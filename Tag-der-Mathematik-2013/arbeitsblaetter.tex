\documentclass[a4paper,ngerman]{scrartcl}

\usepackage[utf8]{inputenc}
\usepackage{amssymb}

\usepackage[ngerman]{babel}
\usepackage{hyperref}

\usepackage{graphicx}

\usepackage[protrusion=true,expansion=true]{microtype}

\usepackage{lmodern}
\usepackage{tabto}

\setlength\parskip{\medskipamount}
\setlength\parindent{0pt}

\usepackage{geometry}
\geometry{tmargin=1.5cm,bmargin=2cm,lmargin=1.5cm,rmargin=1.5cm}

\pagestyle{empty}

\setlength{\fboxrule}{2pt}
\setlength{\fboxsep}{-3pt}

\newcommand{\drawHere}{%
  \begin{center}%
    \fbox{\parbox[c][0.9\textwidth]{0.9\textwidth}{\ }}%
  \end{center}}

\newcommand{\header}{%
  \begin{raggedleft}
  \tiny Universität Augsburg \\
  Tag der Mathematik 2013 \par
  \end{raggedleft}}

\begin{document}

\header

\begin{center}
  \Huge\bf
  Die Kochsche Schneeflocke
\end{center}

\vfill
\drawHere

\vfill
\Large

\renewcommand{\labelitemi}{$\bigstar$}

\begin{itemize}
  \item So konstruiert man die Kochsche Schneeflocke:

  \begin{center}
    \includegraphics[scale=0.5]{koch}
  \end{center}
  \item Was ist ihr Umfang?
  \item Was ist ihr Flächeninhalt?
  \item Dafür ist sie gut: XXX
\end{itemize}

\newpage

\header

\begin{center}
  \Huge\bf
  Das Sierpinski-Dreieck
\end{center}

\vfill
\drawHere

\vfill
\Large

\renewcommand{\labelitemi}{$\blacktriangle$}

\begin{itemize}
  \item Spiele folgendes \emph{Chaosspiel}:
  \begin{enumerate}
    \item Zeichne ein großes Dreieck.
    \item Wähle einen beliebigen Startpunkt im Dreieck.
    \item Such dir zufällig eine der drei Ecken aus.
    \item Markiere als neuen Punkt die Mitte zwischen deiner Ecke und dem
    vorherigen Punkt.
    \item Fahre mit dem neuen Punkt bei Schritt 3 fort.
  \end{enumerate}
  \item Obwohl man den Startpunkt und die Ecken völlig zufällig wählt, ergibt
  sich erstaunlicherweise eine regelmäßige Figur: das \emph{Sierpinski-Dreieck}.
  \item Deterministisch (ohne Zufall) kann man es auch so konstruieren:

  \begin{center}
    \includegraphics[scale=0.1]{sierpinski-1}\hfill
    \includegraphics[scale=0.1]{sierpinski-2}\hfill
    \includegraphics[scale=0.1]{sierpinski-3}\hfill
    \includegraphics[scale=0.1]{sierpinski-4}
  \end{center}
  \item Was ist sein Umfang?
  \item Was ist sein Flächeninhalt?
  \item Dafür ist es gut: XXX
\end{itemize}

\end{document}
