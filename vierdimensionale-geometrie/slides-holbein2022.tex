\documentclass[12pt,compress,ngerman,utf8,t]{beamer}
\usepackage[ngerman]{babel}
\usepackage{calc}
\usepackage{ragged2e,wasysym,multicol,mathtools}
\usepackage[protrusion=true,expansion=true]{microtype}
\usepackage{booktabs}
\usepackage{multimedia}
\hypersetup{colorlinks=true}

\newcommand{\video}[2]{\movie[width=#2,height=#2,autostart,loop,poster]{}{#1}}

\graphicspath{{images/}}

\title[Vierdimensionale Geometrie]{Die wundersame Welt der \\ vierdimensionalen Geometrie}
\author[Ingo Blechschmidt, Alexander Mai]{\scriptsize
\vspace*{-1em} \\
\textbf{Schuljahresendmitmachvortrag am 20. Juli 2022} \\
\emph{Fragen sind jederzeit willkommen! Bitte nicht bis zum Ende aufsparen.} \\
\medskip
Ingo Blechschmidt und Alexander Mai \\
Universität Augsburg}

%\usetheme{Warsaw}
\useinnertheme[shadow=true]{rounded}
\useoutertheme{split}
\usecolortheme{orchid}
\usecolortheme{whale}
\setbeamerfont{block title}{size={}}

\useinnertheme{rectangles}

\usecolortheme{seahorse}
\definecolor{mypurple}{RGB}{150,0,255}
\setbeamercolor{structure}{fg=mypurple}
\definecolor{myred}{RGB}{150,0,0}
\setbeamercolor*{title}{bg=myred,fg=white}
\setbeamercolor*{titlelike}{bg=myred,fg=white}

\usefonttheme{serif}
\usepackage[T1]{fontenc}
\usepackage{libertine}

\setbeamertemplate{navigation symbols}{}

\setbeamertemplate{title page}[default][colsep=-1bp,rounded=false,shadow=false]
\setbeamertemplate{frametitle}[default][colsep=-2bp,rounded=false,shadow=false,center]

\newcommand{\hil}[1]{{\usebeamercolor[fg]{item}{\textbf{#1}}}}
\setbeamertemplate{frametitle}{%
  \vskip1em%
  \leavevmode%
  \begin{beamercolorbox}[dp=1ex,center]{}%
      \usebeamercolor[fg]{item}{\textbf{\textsf{\Large \insertframetitle}}}
  \end{beamercolorbox}%
}

\setbeamertemplate{footline}{%
  \leavevmode%
  \hfill%
  \begin{beamercolorbox}[ht=2.25ex,dp=1ex,right]{}%
    \usebeamerfont{date in head/foot}
    \insertframenumber\,/\,\inserttotalframenumber\hspace*{1ex}
  \end{beamercolorbox}%
  \vskip0pt%
}

\setbeameroption{hide notes}
\setbeamertemplate{note page}[plain]

\begin{document}

\frame{
  \centering
  \includegraphics[width=0.4\textwidth]{great-grand-120-cell}
  \smallskip

  \titlepage
}


\section{Grundlagen}

\begin{frame}[plain,c]
  \centering
  \Large
  \hil{Wie geht's weiter?}

  1, \pause $\infty$, \pause 5, \pause 6, \pause \hil{??}
\end{frame}
\addtocounter{framenumber}{-1}


\subsection{Vier Dimension: was ist das?}

\begin{frame}{Vier Dimensionen?}
  \centering
  \only<1>{\includegraphics[width=0.9\textwidth]{4d-tesseract}}
  \only<2>{\includegraphics[width=0.9\textwidth]{4d-pentachoron}}
  \only<3>{\includegraphics[width=0.9\textwidth]{prisoner}}
  \par
  % Skizze Punkt, Linie, Quadrat, Würfel, nach \pause Tesserakt
  % Skizze Punkt, Linie, Dreieck, Tetraeder, 4-dimensionales Simplex
  % Sache mit Projektion erklären
  % klarstellen, dass wir über vierdimensionalen Raum (nicht Raum+Zeit) sprechen
  % nicht: 11-dimensionaler Quantenschaum
  % Wo erkennt man die drei Dimensionen in der realen Welt?
  % Gefängnisse (Quadrat genügt im Flachland, Würfel genügt in 3D, genügt nicht in 4D)
\end{frame}

\begin{frame}{Kleben vierdimensionaler Formen}
  \centering
  \begin{columns}
    \begin{column}{0.3\textwidth}
      \centering
      \hil{Würfel} \\
      \smallskip
      \includegraphics[height=0.35\textheight]{cube-net}
    \end{column}
    \begin{column}{0.3\textwidth}
      \centering
      \hil{Tesserakt} \\
      \smallskip
      \includegraphics[height=0.35\textheight]{008-cell-net}
    \end{column}
  \end{columns}
  \bigskip
  \bigskip

  \begin{columns}
    \begin{column}{0.3\textwidth}
      \centering
      \hil{16-Zeller} \\
      \smallskip
      \includegraphics[height=0.3\textheight]{016-cell-net}
    \end{column}
    \begin{column}{0.3\textwidth}
      \centering
      \hil{24-Zeller} \\
      \smallskip
      \includegraphics[height=0.3\textheight]{024-cell-net}
    \end{column}
  \end{columns}
\end{frame}


\section[Größen]{Größen in vier Dimensionen}

\subsection{Küssende Hyperkugeln}

\begin{frame}[plain,c]
  \centering\Huge
  \scalebox{2.6}{\hil{Liebe ist}} \\[0.6em]
  \scalebox{2.6}{\hil{wichtig.}}

  \bigskip
  \bigskip

  \scalebox{2.6}{\hil{$\boldsymbol{\heartsuit}$}}
  \par
\end{frame}
\addtocounter{framenumber}{-1}

\begin{frame}{Küssende Hyperkugeln}
  % Bild vier Kreise an den Ecken des Zweiheitsquadrats und Kreis in der Mitte
  \centering
  \vspace*{-1em}
  \includegraphics[width=0.3\textwidth]{sizes-2}
  \bigskip

  {\small
  \begin{tabular}{lll}
    \toprule
    Dimension & Radius der inneren Hyperkugel & \\\midrule
    $n = 2$ & \pause $\sqrt{2} - 1$ \\
    \pause
    $n = 3$ & \pause $\sqrt{3} - 1$ \\
    \pause
    $n = 4$ & $\sqrt{4} - 1$ \\
    \pause
    $n$ & $\sqrt{n} - 1$ \\
    \bottomrule
  \end{tabular}\par}
  \bigskip

  \centering
  \hil{Die Entfernung vom Mittelpunkt zu den Ecken wird immer größer.}
  \par
\end{frame}


\section[Schnitte]{Schnitttheorie}

\subsection{Ankunft eines Hyperballs}

\begin{frame}{Ankunft eines Hyperballs}
  \centering
  \includegraphics[width=0.9\textwidth]{a-hyperball-arrives}
  \par
\end{frame}


\subsection[und eines Tesserakts]{Ankunft eines Tesserakts}

\begin{frame}{Ankunft eines Tesserakts}
  \centering
  \includegraphics[width=0.9\textwidth]{a-tesseract-arrives}
  \par
\end{frame}

% Jetzt Ecken & Co. zählen


\section{Platonische Körper}

\newcommand{\solid}[5]{\begin{column}{0.31\textwidth}\centering\hil{#2}\par#5E, #4K, #3F\\\medskip\includegraphics[height=0.7\textwidth]{#1}\end{column}}
\newcommand{\solidd}[3]{\begin{column}{0.4\textwidth}\centering\hil{#2}\par#3\\\medskip\includegraphics[height=1.9cm]{#1}\end{column}}
\newcommand{\solidde}[3]{\begin{column}{0.33\textwidth}\centering\hil{#2}\par{\scriptsize#3\\}\medskip\includegraphics[height=1.9cm]{#1}\end{column}}
\newcommand{\solidv}[1]{\begin{column}{0.3\textwidth}\centering\hil{$\phantom{A}$}\par$\phantom{A}$\\\medskip\video{images/#1.mp4}{1.9cm}\end{column}}


\subsection{In 3d}

\begin{frame}{Platonische Körper in 3d}
  \begin{columns}[c]
    \solid{tetrahedron}{Tetraeder}{4}{6}{4}
    \solid{hexahedron}{Hexaeder}{6}{12}{8}
    \solid{octahedron}{Oktaeder}{8}{12}{6}
  \end{columns}
  \bigskip
  \begin{columns}[c]
    \solid{dodecahedron}{Dodekaeder}{12}{30}{20}
    \solid{icosahedron}{Ikosaeder}{20}{30}{12}
  \end{columns}
\end{frame}


\subsection{In 4d}

\begin{frame}{Platonische Körper in 4d}
  \only<1>{
    \begin{columns}[c]
      \solidd{tetrahedron}{Tetraeder}{4E, 6K, 4F}
      \solidd{005-cell}{Pentachoron}{5E, 10K, 10F, 5Z}
      \solidv{005-cell}
    \end{columns}
  }

  \only<2>{
    \begin{columns}[c]
      \solidd{hexahedron}{Hexaeder}{8E, 12K, 6F}
      \solidd{008-cell}{Octachoron}{16E, 32K, 24F, 8Z}
      \solidv{008-cell}
    \end{columns}
    \vspace*{2em}
    \begin{columns}[c]
      \solidd{octahedron}{Oktaeder}{6E, 12K, 8F}
      \solidd{016-cell}{Hexadecachoron}{8E, 24K, 32F, 16Z}
      \solidv{016-cell}
    \end{columns}
  }

  \only<3>{
    \begin{columns}[c]
      \solidd{dodecahedron}{Dodekaeder}{20E, 30K, 12F}
      \solidd{120-cell}{Hecatonicosachoron}{600E, 1200K, 720F, 120Z}
      \solidv{120-cell}
    \end{columns}
    \vspace*{2em}
    \begin{columns}[c]
      \solidd{icosahedron}{Ikosaeder}{12E, 30K, 20F}
      \solidd{600-cell}{Hexacosichoron}{120E, 720K, 1200F, 600Z}
      \solidv{600-cell}
    \end{columns}
  }

  \only<4-5>{
    \centering
    \hil{Icositetrachoron}\par
    24E, 96K, 96F, 24Z\\\medskip
    \only<4>{\includegraphics[width=0.5\textwidth]{024-cell}}
    \only<5>{\video{images/024-cell.mp4}{0.5\textwidth}}
  }
\end{frame}

\begin{frame}{Pflasterungen}
  \centering
  \includegraphics[height=0.6\textheight]{tesselation-squares}
  \qquad
  \includegraphics[height=0.6\textheight]{hyperdiamond-3d}
  \bigskip

  \hil{Der 24-Zeller pflastert den vierdimensionalen Raum.}
\end{frame}

\begin{frame}{Übersicht}
  \begin{columns}[c]
    \solidde{005-cell}{Pentachoron}{5E, 10K, 10F, 5Z}
    \solidde{008-cell}{Octachoron}{16E, 32K, 24F, 8Z}
    \solidde{016-cell}{Hexadecachoron}{8E, 24K, 32F, 16Z}
  \end{columns}
  \bigskip
  \bigskip
  \begin{columns}[c]
    \solidde{120-cell}{Hecatonicosachoron}{600E, 1200K, 720F, 120Z}
    \solidde{600-cell}{Hexacosichoron}{120E, 720K, 1200F, 600Z}
    \solidde{024-cell}{Icositetrachoron}{24E, 96K, 96F, 24Z}
  \end{columns}
\end{frame}


\begin{frame}{Ein vierdimensionales Labyrinth}
  \includegraphics[width=\textwidth]{4d-labyrinth}
\end{frame}


\subsection{Ausblick}

\begin{frame}{Die vierte Dimension \ldots}
  \begin{enumerate}
    \item ist faszinierend schön,

    \item hilft beim Verständnis der dritten Dimension,

    \begin{center}
      \includegraphics[height=0.3\textheight]{torus-circles}
      \qquad\qquad
      \includegraphics[height=0.3\textheight]{hopf-fibration}
    \end{center}

    \item ist unabdingbar für die moderne Physik und

    \item ist als einzige Dimension noch größtenteils unverstanden.

    \begin{center}
      \begin{tabular}{l|rrrrrrrrrr}
        Dimension & 1 & 2 & 3 & 4 & 5 & 6 & 7 & 8 & 9 & \ldots \\
        Anzahl Sphären & 1 & 1 & 1 & \hil{??} & 1 & 1 & 28 & 2 & 8 & \ldots
      \end{tabular}
    \end{center}
  \end{enumerate}
\end{frame}

\begin{frame}[plain,c]
  \centering
  \begin{columns}
    \begin{column}{0.4\textwidth}
      \centering
      \hil{Catharina Stroppel} \\
      $\phantom{\text{g}}$Knotentheoretikerin$\phantom{\text{g}}$ \\
      \smallskip
      \includegraphics[height=0.5\textheight]{catharina-stroppel}
    \end{column}
    \begin{column}{0.6\textwidth}
      \centering
      \hil{Julia Grigsby} \\
      niedrigdimensionale Topologin \\
      \smallskip
      \includegraphics[height=0.5\textheight]{julia-grigsby}
    \end{column}
  \end{columns}
\end{frame}
\addtocounter{framenumber}{-1}

\begin{frame}[plain]
  \centering
  \medskip
  \includegraphics[height=0.95\textheight]{salvador-dali} \\
  \scriptsize
  Salvador Dalí: \hil{Corpus Hypercubus} (1954)
\end{frame}
\addtocounter{framenumber}{-1}

\appendix

\section{Image sources}

\begin{frame}{Image sources}
  \tiny

  \only<1>{Rendered images of four-dimensional bodies created by Robert Webb with his
  Stella software: \\
  \url{https://en.wikipedia.org/wiki/File:Ortho_solid_011-uniform_polychoron_53p-t0.png} \\
  \url{https://en.wikipedia.org/wiki/File:Schlegel_wireframe_5-cell.png} \\
  \url{https://en.wikipedia.org/wiki/File:Schlegel_wireframe_8-cell.png} \\
  \url{https://en.wikipedia.org/wiki/File:Schlegel_wireframe_16-cell.png} \\
  \url{https://en.wikipedia.org/wiki/File:Schlegel_wireframe_24-cell.png} \\
  \url{https://en.wikipedia.org/wiki/File:Schlegel_wireframe_120-cell.png} \\
  \url{https://en.wikipedia.org/wiki/File:Schlegel_wireframe_600-cell_vertex-centered.png}}

  \only<2>{Miscellaneous pictures: \\
  \url{http://4.bp.blogspot.com/_TbkIC-eqFNM/S-dK9dd1cUI/AAAAAAAAFjA/d8qdTHhKy1U/s320/tesseract+unfolded.png} \\
  \url{http://gwydir.demon.co.uk/jo/tess/optical6.gif}
  \url{https://commons.wikimedia.org/wiki/File:Blue_Trefoil_Knot.png} \\
  \url{https://en.wikipedia.org/wiki/File:Dodecahedron.svg} \\
  \url{https://en.wikipedia.org/wiki/File:Hexahedron.svg} \\
  \url{https://en.wikipedia.org/wiki/File:Icosahedron.svg} \\
  \url{https://en.wikipedia.org/wiki/File:Octahedron.svg} \\
  \url{https://en.wikipedia.org/wiki/File:Tetrahedron.svg} \\
  \url{https://mathlesstraveled.files.wordpress.com/2017/01/villarceau-torus-small.jpg} \\
  \url{https://upload.wikimedia.org/wikipedia/commons/1/1e/600-cell.gif} \\
  \url{https://upload.wikimedia.org/wikipedia/commons/2/24/HC_R1.png} \\
  \url{https://upload.wikimedia.org/wikipedia/commons/7/72/Rhombic_dodecahedra_b.png} \\
  \url{https://upload.wikimedia.org/wikipedia/commons/a/a0/16-cell.gif} \\
  \url{https://upload.wikimedia.org/wikipedia/commons/c/cf/Hexahedron_flat_color.svg} \\
  \url{https://upload.wikimedia.org/wikipedia/commons/d/d6/8-cell-orig.gif} \\
  \url{https://upload.wikimedia.org/wikipedia/commons/d/d8/5-cell.gif} \\
  \url{https://upload.wikimedia.org/wikipedia/commons/f/f4/24-cell.gif} \\
  \url{https://upload.wikimedia.org/wikipedia/commons/f/f9/120-cell.gif} \\
  \url{https://upload.wikimedia.org/wikipedia/commons/thumb/b/b9/Hopf_Fibration.png/250px-Hopf_Fibration.png} \\
  \url{https://upload.wikimedia.org/wikipedia/en/0/09/Dali_Crucifixion_hypercube.jpg} \\
  \url{https://www2.bc.edu/julia-grigsby/Eli_Moab_6in.JPG} \\
  \url{http://www.gnuplotting.org/figs/klein_bottle.png} \\
  \url{http://www.math.uni-bonn.de/ag/stroppel/Picture_cs2.jpg}}
\end{frame}
\addtocounter{framenumber}{-1}

\end{document}


Ecken:     2 4  8 16
Kanten:    1 4 12 32    32 = (16 * 4) / 2    12 = (8 * 3) / 2
Flächen:   0 1  6 24    (3über2)*8/4 = 6     (4über2)*16/4 = 24 = 2^2 * (4über2)
Volumina:  0 0  1  8
4-Vol.:    0 0  0  1

* Raumzeit in 2+1 Dimensionen ist schlecht wegen: Feldgleichung hat zu wenig
  Unbekannte. Masse müsste immer Null sein. (?)
* Kommutativität von gewissen Rotationen; Verknüpfungen von Rotationen ist nicht
  unbedingt Rotation (einer Ebene)! (?)
* (Rotationen nicht um Achsen, sondern "um Ebenen"; besser: Man rotiert immer "in
  einer Ebene".)
* Quaternionen?
* Ist jedes Element der SO(4) eine Rotation einer Ebene? Nein. (-Id)
* Ist jedes Element der SO(4) Verknüpfung zweier Spiegelungen an Hyperebenen? Nein.

Man könnte den Vortrag auch mit dem bekannten Witz (Mathematikerin, Physikerin,
Konferenz, Schwierigkeit, ganz einfach: dann n gegen 11) beginnen. Und dann
gleich dazu sagen, dass es bei uns um räumliche Dimensionen geht.
