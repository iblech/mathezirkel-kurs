\documentclass[12pt,compress,ngerman,utf8,t]{beamer}
\usepackage[ngerman]{babel}
\usepackage{calc}
\usepackage{ragged2e,wasysym,multicol,mathtools}
\usepackage[protrusion=true,expansion=true]{microtype}
\usepackage{booktabs}
\usepackage{multimedia}
\hypersetup{colorlinks=true}

\newcommand{\video}[2]{\movie[width=#2,height=#2,autostart,loop,poster]{}{#1}}

\graphicspath{{images/}}

\title[Vierdimensionale Geometrie]{Die wundersame Welt der \\ vierdimensionalen Geometrie}
\author[Ingo Blechschmidt, Matthias Hutzler]{\scriptsize
\vspace*{-1em} \\
\textbf{Weihnachtsvorlesung am 18. Dezember 2017} \\
\emph{Fragen sind jederzeit willkommen! Bitte nicht bis zum Ende aufsparen.} \\
\medskip
Ingo Blechschmidt und Matthias Hutzler \\
Lehrstuhl für Algebra und Zahlentheorie}

%\usetheme{Warsaw}
\useinnertheme[shadow=true]{rounded}
\useoutertheme{split}
\usecolortheme{orchid}
\usecolortheme{whale}
\setbeamerfont{block title}{size={}}

\useinnertheme{rectangles}

\usecolortheme{seahorse}
\definecolor{mypurple}{RGB}{150,0,255}
\setbeamercolor{structure}{fg=mypurple}
\definecolor{myred}{RGB}{150,0,0}
\setbeamercolor*{title}{bg=myred,fg=white}
\setbeamercolor*{titlelike}{bg=myred,fg=white}

\usefonttheme{serif}
\usepackage[T1]{fontenc}
\usepackage{libertine}

\setbeamertemplate{navigation symbols}{}

\setbeamertemplate{title page}[default][colsep=-1bp,rounded=false,shadow=false]
\setbeamertemplate{frametitle}[default][colsep=-2bp,rounded=false,shadow=false,center]

\newcommand{\hil}[1]{{\usebeamercolor[fg]{item}{\textbf{#1}}}}
\setbeamertemplate{frametitle}{%
  \vskip1em%
  \leavevmode%
  \begin{beamercolorbox}[dp=1ex,center]{}%
      \usebeamercolor[fg]{item}{\textbf{\textsf{\Large \insertframetitle}}}
  \end{beamercolorbox}%
}

\setbeamertemplate{footline}{%
  \leavevmode%
  \hfill%
  \begin{beamercolorbox}[ht=2.25ex,dp=1ex,right]{}%
    \usebeamerfont{date in head/foot}
    \insertframenumber\,/\,\inserttotalframenumber\hspace*{1ex}
  \end{beamercolorbox}%
  \vskip0pt%
}

\setbeameroption{hide notes}
\setbeamertemplate{note page}[plain]

\begin{document}

\frame{
  \centering
  \includegraphics[width=0.4\textwidth]{great-grand-120-cell}
  \smallskip

  \titlepage
}

\section{Grundlagen}

\subsection{Vier Dimension: was ist das?}

\begin{frame}{Vier Dimensionen?}
  \centering
  \only<1>{\includegraphics[width=0.8\textwidth]{4d-tesseract-plain}}
  \only<2>{\includegraphics[width=0.9\textwidth]{4d-pentachoron}}
  \only<3>{\includegraphics[width=0.9\textwidth]{prisoner}}
  \par
  % Skizze Punkt, Linie, Quadrat, Würfel, nach \pause Tesserakt
  % Skizze Punkt, Linie, Dreieck, Tetraeder, 4-dimensionales Simplex
  % Sache mit Projektion erklären
  % klarstellen, dass wir über vierdimensionalen Raum (nicht Raum+Zeit) sprechen
  % nicht: 11-dimensionaler Quantenschaum
  % Wo erkennt man die drei Dimensionen in der realen Welt?
  % Gefängnisse (Quadrat genügt im Flachland, Würfel genügt in 3D, genügt nicht in 4D)
\end{frame}

\note{
  \begin{itemize}
    \item On the previous slide you see two-dimensional projections of the
    three-dimensional cube and the four-dimensional hypercube (tesseract).
    \item We're talking about four spatial dimensions. This is not related to
    four-dimensional spacetime or eleven-dimensional string theory.
    \item A flatlander can be imprisoned by enclosing them with a square. But
    we, as three-dimensional beings, can free them by grabbing them, lifting
    them up in the third dimension, moving them a little to the side, and
    putting them back into flatland.
    \item Similarly, a four-dimensional being could free us if we were imprisoned
    in a three-dimensional cube.
  \end{itemize}
}


\subsection{Knotentheorie}

\begin{frame}{Schnürsenkel binden}
  % Erklären, dass Schnürsenkel immer aufgehen würden
  \centering
  \includegraphics[width=0.65\textwidth]{trefoil-knot}
  \par
\end{frame}

\note{
  \begin{itemize}
    \item You can untie any knot in four dimensions. Two linked one-dimensional
    strings can always be separated in four dimensions.
    \item But it's possible to tangle an one-dimensional string with the
    two-dimensional surface of a sphere in four dimensions.
    \item More generally, in $n$ dimensions, one can tangle $a$-dimensional
    objects with $b$-dimensional objects provided that $a + b \geq n - 1$.
  \end{itemize}
}

%Verhedderbar
%in 3D: Schnur, Schnur; 2, 2; 1, 1
%in 3D: Punkt, Fläche;  0, 1; 0, 2
%in 2D: Schnur, Punkt;  1, 2; 1, 0
%in 4D: Schnur, Fläche; 3, 2; 1, 2
%
%Beh.: Dimensionen addiert muss >= n - 1 sein, damit Verhedderung möglich ist
%
%Nicht verhedderbar
%in 3D: Punkt, Schnur; 3, 2; 0, 1
%in 3D: Punkt, Punkt;  3, 3; 0, 0
%in 4D: Punkt, Fläche      ; 0, 2
%in 4D: Schnur, Schnur     ; 1, 1


\section[Größen]{Größen in vier Dimensionen}

\subsection{Volumenverhältnisse}

\begin{frame}{Hypervolumen von Hyperkugeln}
  \centering
  \vspace*{-0.5em}
  \includegraphics[width=0.4\textwidth]{sizes-1}
  \bigskip
  % vorher gscheite Definition der Hyperkugel
  % Bild vom Kreis im Quadrat
  % Bild vom Zylinder im Würfel
  % Bild von der Kugel im Würfel

  \small
  \begin{tabular}{lll}
    \toprule
    Dimension & Hypervolumen & \\\midrule
    $n = 2$ & $\pi / 4$ & $\approx 0{,}785 \,\mathrm{m}^2$ \\
    $n = 3$ & $\pi / 6$ & $\approx 0{,}524 \,\mathrm{m}^3$ \\
    $n = 4$ & $\pi^2 / 32$ & $\approx 0{,}308 \,\mathrm{m}^4$ \\
    $n = 5$ & $\pi^2 / 60$ & $\approx 0{,}164 \,\mathrm{m}^5$ \\
    $n = 6$ & $\pi^3 / 384$ & $\approx 0{,}081 \,\mathrm{m}^6$ \\
    $n = 7$ & $\pi^3 / 840$ & $\approx 0{,}037 \,\mathrm{m}^7$ \\
    $n \to \infty$ & $\to 0$ \\
    \bottomrule
  \end{tabular}\par
\end{frame}

\note{
  \begin{itemize}
    \item The portion of the $n$-dimensional unit hypercube which is occupied
    by the inscribed $n$-dimensional hyperball gets arbitrary small in
    sufficiently high dimensions.
    \item The volume of such a hyperball is the answer to the following
    question: What is the probability that we managed to hit the hyperball with
    an dart, provided that we managed to hit the enclosing hyperball?
    \item Wikipedia gives
    \href{https://en.wikipedia.org/wiki/Volume_of_an_n-ball}{derivations for
    these formulas}.
    \item You can use the \emph{power of negative thinking} to motivate that
    the formula for the n-dimensional volume of the n-dimensional hyperball
    does \emph{not} contain $\pi^n$ (but rather $\pi^{\lfloor n/2 \rfloor}$):
    Think about the zero- and one-dimensional case.

    A zero-dimensional ball is just a point. Its zero-dimensional volume
    is~$1$.

    An one-dimensional ball is just a line segment. Its one-dimensional volume
    is its length.
  \end{itemize}
}


\subsection{Küssende Hyperkugeln}

\begin{frame}[plain,c]
  \centering\Huge
  \scalebox{2.6}{\hil{Liebe ist}} \\[0.6em]
  \scalebox{2.6}{\hil{wichtig.}}

  \bigskip
  \bigskip

  \scalebox{2.6}{\hil{$\boldsymbol{\heartsuit}$}}
  \par
\end{frame}
\addtocounter{framenumber}{-1}

\begin{frame}{Küssende Hyperkugeln}
  % Bild vier Kreise an den Ecken des Zweiheitsquadrats und Kreis in der Mitte
  \centering
  \vspace*{-1em}
  \includegraphics[width=0.3\textwidth]{sizes-2}
  \bigskip

  {\small
  \begin{tabular}{lll}
    \toprule
    Dimension & Radius der inneren Hyperkugel & \\\midrule
    $n = 2$ & \pause $\sqrt{2} - 1$ \\
    \pause
    $n = 3$ & \pause $\sqrt{3} - 1$ \\
    \pause
    $n = 4$ & $\sqrt{4} - 1$ \\
    \pause
    $n$ & $\sqrt{n} - 1$ \\
    \bottomrule
  \end{tabular}\par}
  \bigskip

  \centering
  \hil{Die Entfernung zu den Ecken wird immer größer.}
  \par
\end{frame}

\note{
  \begin{itemize}
    \item In two dimensions, the distance of a point~$(x,y)$ to the origin is
    $\sqrt{x^2 + y^2}$ (by the Pythagorean theorem).
    \item In three dimensions, the distance of a point~$(x,y,z)$ to the origin
    is~$\sqrt{x^2+y^2+z^2}$.
    \item The pattern continues to arbitrary dimensions.
    \item In four dimensions, the ``small hypersphere in the middle'' has
    exactly the same size as the hyperspheres at the 16 vertices of the
    hypercube.
    \item In even greater dimensions, the hyperspheres at the vertices are so
    small that the ``small hypersphere in the middle'' is bigger than them and
    in fact bigger than the hypercube!
  \end{itemize}
}


\subsection[Relativität]{Allgemeine Relativitätstheorie}

\begin{frame}{Allgemeine Relativitätstheorie}
  \vspace*{-1em}
  \begin{center}
    \includegraphics[height=0.3\textheight]{einstein}
    \qquad
    \includegraphics[height=0.3\textheight]{gravitational-waves}
  \end{center}

  Einsteins gefeierte \hil{Feldgleichung} besagt
  \[ G = \kappa \cdot T, \]
  wobei
  \begin{itemize}
    \item $G$ die \hil{Raumkrümmung} angibt,
    \item $T$ die \hil{Massenverteilung} misst und
    \item $\kappa$ eine Konstante ist.
  \end{itemize}

  In $2+1$ Dimensionen impliziert die Gleichung~$T = 0$.
  Nur in vier und mehr Dimensionen ist die Theorie nichttrivial.
\end{frame}

\note{
  Details are in the article
  \href{http://www.csun.edu/~vcphy00d/PDFPublications/1977\%20GR(2+1).pdf}{General
  relativity in two and three-dimensional space-times} by Peter Collas.
}


\section[Schnitte]{Schnitttheorie}

\subsection{Ankunft eines Hyperballs}

\begin{frame}{Ankunft eines Hyperballs}
  \centering
  \includegraphics[width=0.9\textwidth]{a-hyperball-arrives}
  \par
\end{frame}


\subsection[und eines Tesserakts]{Ankunft eines Tesserakts}

\begin{frame}{Ankunft eines Tesserakts}
  \centering
  \includegraphics[width=0.9\textwidth]{a-tesseract-arrives}
  \par
\end{frame}

% Jetzt Ecken & Co. zählen


\subsection{Ein 4d Fraktal}

\begin{frame}{Ein vierdimensionales Fraktal}
  \justifying
  Ihr kennt das Mandelbrotfraktal.
  Vielleicht kennt ihr auch die Juliafraktale, von denen es je eins für jeden
  Punkt der Ebene gibt.
  Aber wusstet ihr, dass diese unendlich vielen Fraktale nur zweidimensionale
  Schnitte eines vereinheitlichten vierdimensionalen Fraktals ist?
  Wir laden euch ein,
  \href{https://rawgit.com/MatthiasHu/FractalsWebGL/4d/page.html}{damit zu spielen}.
  \bigskip

  \centering
  \includegraphics[width=0.5\textwidth]{mandelbrot}
  \par
\end{frame}


\section{Platonische Körper}

\newcommand{\solid}[5]{\begin{column}{0.31\textwidth}\centering\hil{#2}\par#5 E, #4 K, #3 F\\\medskip\includegraphics[height=0.7\textwidth]{#1}\end{column}}
\newcommand{\solidd}[3]{\begin{column}{0.4\textwidth}\centering\hil{#2}\par#3\\\medskip\includegraphics[height=1.9cm]{#1}\end{column}}
\newcommand{\solidde}[3]{\begin{column}{0.33\textwidth}\centering\hil{#2}\par{\scriptsize#3\\}\medskip\includegraphics[height=1.9cm]{#1}\end{column}}
\newcommand{\solidv}[1]{\begin{column}{0.3\textwidth}\centering\hil{$\phantom{A}$}\par$\phantom{A}$\\\medskip\video{images/#1.mp4}{1.9cm}\end{column}}


\subsection{In 3d}

\begin{frame}{Platonische Körper in 3d}
  \begin{columns}[c]
    \solid{tetrahedron}{Tetraeder}{4}{6}{4}
    \solid{hexahedron}{Hexaeder}{6}{12}{8}
    \solid{octahedron}{Oktaeder}{8}{12}{6}
  \end{columns}
  \bigskip
  \begin{columns}[c]
    \solid{dodecahedron}{Dodekaeder}{12}{30}{20}
    \solid{icosahedron}{Ikosaeder}{20}{30}{12}
  \end{columns}
\end{frame}


\subsection{In 4d}

\begin{frame}{Platonische Körper in 4d}
  \only<1>{
    \begin{columns}[c]
      \solidd{tetrahedron}{Tetraeder}{4E, 6K, 4f}
      \solidd{005-cell}{Pentachoron}{5E, 10K, 10F, 5Z}
      \solidv{005-cell}
    \end{columns}
  }

  \only<2>{
    \begin{columns}[c]
      \solidd{hexahedron}{Hexaeder}{8E, 12K, 6f}
      \solidd{008-cell}{Octachoron}{16E, 32K, 24F, 8Z}
      \solidv{008-cell}
    \end{columns}
    \vspace*{2em}
    \begin{columns}[c]
      \solidd{octahedron}{Oktaeder}{6K, 12K, 8f}
      \solidd{016-cell}{Hexadecachoron}{8E, 24K, 32F, 16Z}
      \solidv{016-cell}
    \end{columns}
  }

  \only<3>{
    \begin{columns}[c]
      \solidd{dodecahedron}{Dodekaeder}{20E, 30K, 12f}
      \solidd{120-cell}{Hecatonicosachoron}{600E, 1200K, 720F, 120Z}
      \solidv{120-cell}
    \end{columns}
    \vspace*{2em}
    \begin{columns}[c]
      \solidd{icosahedron}{Ikosaeder}{12E, 30K, 20f}
      \solidd{600-cell}{Hexacosichoron}{120E, 720K, 1200F, 600Z}
      \solidv{600-cell}
    \end{columns}
  }

  \only<4-5>{
    \centering
    \hil{Icositetrachoron}\par
    24E, 96K, 96F, 24Z\\\medskip
    \only<4>{\includegraphics[width=0.5\textwidth]{024-cell}}
    \only<5>{\video{images/024-cell.mp4}{0.5\textwidth}}
  }
\end{frame}

\begin{frame}{Pflasterungen}
  \centering
  \includegraphics[height=0.6\textheight]{tesselation-squares}
  \qquad
  \includegraphics[height=0.6\textheight]{hyperdiamond-3d}
  \bigskip

  \hil{Der 24-Zeller pflastert den vierdimensionalen Raum.}
\end{frame}

\begin{frame}{Übersicht}
  \begin{columns}[c]
    \solidde{005-cell}{Pentachoron}{5E, 10K, 10F, 5Z}
    \solidde{008-cell}{Octachoron}{16E, 32K, 24F, 8Z}
    \solidde{016-cell}{Hexadecachoron}{8E, 24K, 32F, 16Z}
  \end{columns}
  \bigskip
  \bigskip
  \begin{columns}[c]
    \solidde{120-cell}{Hecatonicosachoron}{600E, 1200K, 720F, 120Z}
    \solidde{600-cell}{Hexacosichoron}{120E, 720K, 1200F, 600Z}
    \solidde{024-cell}{Icositetrachoron}{24E, 96K, 96F, 24Z}
  \end{columns}
\end{frame}


%\subsection{In beliebigen Dimensionen}
%
%\begin{frame}{In beliebigen Dimensionen}
%  \centering
%  \begin{tabular}{ll}
%    \toprule
%    Dimension & Anzahl platonischer Körper \\ \midrule
%    $n = 1$ & 1 (nur die Linie) \\
%    $n = 2$ & $\infty$ (Dreieck, Quadrat, Fünfeck, Sechseck, \ldots) \\
%    $n = 3$ & 5 \\
%    $n = 4$ & 6 \\
%    $n = 5$ & 3 (nur Simplex, Hyperwürfel und sein Duales) \\
%    $n = 6$ & 3 (nur Simplex, Hyperwürfel und sein Duales) \\
%    $n = 7$ & 3 (nur Simplex, Hyperwürfel und sein Duales) \\
%    $n = 8$ & 3 (nur Simplex, Hyperwürfel und sein Duales) \\
%    \emph{und so weiter} \\
%    \bottomrule
%  \end{tabular}
%  \par
%\end{frame}

\note{
  \begin{itemize}
    \item The only platonic solid which can be used to tesselate
    three-dimensional space is the cube.
    \item In four dimensions, both the tesseract and the 24-cell work.
    \item This has a deeper reason: In any dimension~$n$, the $n$-dimensional
    analogue of the rhombic dodecahedron can be used to tesselate
    $n$-dimensional space. In dimension $n = 3$ the rhombic dodecahedron is not
    a Platonic solid; in dimension $n = 4$ it is (and is also called the
    ``24-cell'').
  \end{itemize}
}


\subsection[Verkleben]{Zusammenkleben vierdimensionaler Formen}

\begin{frame}{Kleben vierdimensionaler Formen}
  \centering
  \begin{columns}
    \begin{column}{0.3\textwidth}
      \centering
      \hil{Würfel} \\
      \smallskip
      \includegraphics[height=0.35\textheight]{cube-net}
    \end{column}
    \begin{column}{0.3\textwidth}
      \centering
      \hil{Tesserakt} \\
      \smallskip
      \includegraphics[height=0.35\textheight]{008-cell-net}
    \end{column}
  \end{columns}
  \bigskip
  \bigskip

  \begin{columns}
    \begin{column}{0.3\textwidth}
      \centering
      \hil{16-Zeller} \\
      \smallskip
      \includegraphics[height=0.3\textheight]{016-cell-net}
    \end{column}
    \begin{column}{0.3\textwidth}
      \centering
      \hil{24-Zeller} \\
      \smallskip
      \includegraphics[height=0.3\textheight]{024-cell-net}
    \end{column}
  \end{columns}
\end{frame}

\begin{frame}[plain]
  \centering
  \medskip
  \includegraphics[height=0.95\textheight]{salvador-dali} \\
  \scriptsize
  Salvador Dalí: \hil{Corpus Hypercubus} (1954)
\end{frame}

\begin{frame}{Ein vierdimensionales Labyrinth}
  \includegraphics[width=\textwidth]{4d-labyrinth}
\end{frame}


\subsection{Ausblick}

\begin{frame}{Die vierte Dimension \ldots}
  \begin{enumerate}
    \item ist faszinierend schön,

    \item hilft beim Verständnis der dritten Dimension,

    \begin{center}
      \includegraphics[height=0.3\textheight]{torus-circles}
      \qquad\qquad
      \includegraphics[height=0.3\textheight]{hopf-fibration}
    \end{center}

    \item ist unabdingbar für die moderne Physik und

    \item ist als einzige Dimension noch größtenteils unverstanden.

    \begin{center}
      \begin{tabular}{l|rrrrrrrrrr}
        Dimension & 1 & 2 & 3 & 4 & 5 & 6 & 7 & 8 & 9 & \ldots \\
        Anzahl Sphären & 1 & 1 & 1 & \hil{??} & 1 & 1 & 28 & 2 & 8 & \ldots
      \end{tabular}
    \end{center}
  \end{enumerate}
\end{frame}

\begin{frame}[plain,c]
  \centering
  \begin{columns}
    \begin{column}{0.4\textwidth}
      \centering
      \hil{Catharina Stroppel} \\
      $\phantom{\text{g}}$Knotentheoretikerin$\phantom{\text{g}}$ \\
      \smallskip
      \includegraphics[height=0.5\textheight]{catharina-stroppel}
    \end{column}
    \begin{column}{0.6\textwidth}
      \centering
      \hil{Julia Grigsby} \\
      niedrigdimensionale Topologin \\
      \smallskip
      \includegraphics[height=0.5\textheight]{julia-grigsby}
    \end{column}
  \end{columns}
\end{frame}

\newcommand{\hero}[2]{\hil{#2}\\\smallskip\includegraphics[height=4cm]{#1}}
\newcommand{\specialhero}[2]{\hil{#2}\\\smallskip\includegraphics[height=5cm]{#1}}
\newcommand{\two}[2]{
  \begin{columns}
    \begin{column}{0.4\textwidth}
      \centering
      #1
    \end{column}
    \begin{column}{0.4\textwidth}
      \centering
      #2
    \end{column}
  \end{columns}
}

\begin{frame}{Applaus für unsere Helden!}
  \centering
  \only<1>{\hero{005-cell}{Pentachoron}}
  \only<2>{\two{\hero{008-cell}{Tesserakt}}{\hero{016-cell}{Hexadecachoron}}}
  \only<3>{\two{\hero{120-cell}{Hecatonicosachoron}}{\hero{600-cell}{Hexacosichoron}}}
  \only<4->{\specialhero{024-cell}{Icositetrachoron}}
  \bigskip

  \only<5>{
    \hil{https://4d.speicherleck.de/}
  }
\end{frame}

\appendix

\section{Image sources}

\begin{frame}{Image sources}
  \tiny
  Miscellaneous pictures: \\
  \url{https://commons.wikimedia.org/wiki/File:Blue_Trefoil_Knot.png} \\
  \url{http://www.gnuplotting.org/figs/klein_bottle.png} \\
  \url{http://4.bp.blogspot.com/_TbkIC-eqFNM/S-dK9dd1cUI/AAAAAAAAFjA/d8qdTHhKy1U/s320/tesseract+unfolded.png} \\
  \url{https://en.wikipedia.org/wiki/File:Tetrahedron.svg} \\
  \url{https://en.wikipedia.org/wiki/File:Hexahedron.svg} \\
  \url{https://en.wikipedia.org/wiki/File:Octahedron.svg} \\
  \url{https://en.wikipedia.org/wiki/File:Dodecahedron.svg} \\
  \url{https://en.wikipedia.org/wiki/File:Icosahedron.svg}
  \bigskip

  Rendered images of four-dimensional bodies created by Robert Webb with his
  Stella software: \\
  \url{https://en.wikipedia.org/wiki/File:Ortho_solid_011-uniform_polychoron_53p-t0.png} \\
  \url{https://en.wikipedia.org/wiki/File:Schlegel_wireframe_5-cell.png} \\
  \url{https://en.wikipedia.org/wiki/File:Schlegel_wireframe_8-cell.png} \\
  \url{https://en.wikipedia.org/wiki/File:Schlegel_wireframe_16-cell.png} \\
  \url{https://en.wikipedia.org/wiki/File:Schlegel_wireframe_24-cell.png} \\
  \url{https://en.wikipedia.org/wiki/File:Schlegel_wireframe_120-cell.png} \\
  \url{https://en.wikipedia.org/wiki/File:Schlegel_wireframe_600-cell_vertex-centered.png} \\
\end{frame}
\addtocounter{framenumber}{-1}

\end{document}


Ecken:     2 4  8 16
Kanten:    1 4 12 32    32 = (16 * 4) / 2    12 = (8 * 3) / 2
Flächen:   0 1  6 24    (3über2)*8/4 = 6     (4über2)*16/4 = 24 = 2^2 * (4über2)
Volumina:  0 0  1  8
4-Vol.:    0 0  0  1

* Raumzeit in 2+1 Dimensionen ist schlecht wegen: Feldgleichung hat zu wenig
  Unbekannte. Masse müsste immer Null sein. (?)
* Kommutativität von gewissen Rotationen; Verknüpfungen von Rotationen ist nicht
  unbedingt Rotation (einer Ebene)! (?)
* (Rotationen nicht um Achsen, sondern "um Ebenen"; besser: Man rotiert immer "in
  einer Ebene".)
* Quaternionen?
* Ist jedes Element der SO(4) eine Rotation einer Ebene? Nein. (-Id)
* Ist jedes Element der SO(4) Verknüpfung zweier Spiegelungen an Hyperebenen? Nein.

Man könnte den Vortrag auch mit dem bekannten Witz (Mathematikerin, Physikerin,
Konferenz, Schwierigkeit, ganz einfach: dann n gegen 11) beginnen. Und dann
gleich dazu sagen, dass es bei uns um räumliche Dimensionen geht.
