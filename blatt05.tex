\documentclass{zirkelblatt}
\geometry{tmargin=2cm,bmargin=3cm,lmargin=3cm,rmargin=3cm}
\usepackage{empheq}
\usepackage{float}
\floatstyle{ruled}
\restylefloat{figure}
\newcommand{\head}[1]{\section*{\rmfamily #1}}%begin{center}\large \textbf{#1}\end{center}}
\let\raggedsection\centering
\newcommand{\ZZ}{\mathbb{Z}}
\newcommand{\RR}{\mathbb{R}}
\newcommand{\xra}[1]{\xrightarrow{#1}}

\theoremstyle{definition}
\newtheorem{defn}{Definition}[section]
\newtheorem{axiom}[defn]{Axiom}
\newtheorem{bsp}[defn]{Beispiel}

\theoremstyle{plain}

\newtheorem{prop}[defn]{Proposition}
\newtheorem{motto}[defn]{Motto}
\newtheorem{wunder}[defn]{Wunder}
\newtheorem{ueberlegung}[defn]{Überlegung}
\newtheorem{lemma}[defn]{Lemma}
\newtheorem{kor}[defn]{Korollar}
\newtheorem{hilfsaussage}[defn]{Hilfsaussage}
\newtheorem{satz}[defn]{Satz}

\theoremstyle{remark}
\newtheorem{bem}[defn]{Bemerkung}
\newtheorem{aufg}[defn]{Aufgabe}

\begin{document}

\maketitleSF{Klasse 11./12., Gruppe 2}{31. Mai 2014: \\ Synthetische
Differentialgeometrie}

\section{Einleitung}

Eine \emph{infinitesimale Zahl}~$\varepsilon$ ist eine Zahl, deren Quadrat Null
ist:~$\varepsilon^2 = 0$. In der gewöhnlichen mathematischen Welt gibt es nur
eine einzige infinitesimale Zahl, nämlich die Zahl~$0$. Für gewisse Anwendungen wäre es
aber schön, wenn es auch interessantere infinitesimale Zahlen gäbe. Ähnlich wie
bei den komplexen Zahlen kann man solche Zahlen \emph{künstlich} konstruieren;
anders als bei den komplexen Zahlen muss man dafür aber gleich ein ganzes neues
\emph{mathematisches Universum} errichten.

Das Teilgebiet der Mathematik, in dem man solche Universen studiert, heißt
\emph{synthetische Differentialgeometrie} und ist ein relativ junges
Forschungsgebiet. Erste Ideen gehen auf die alten Griechen und Versuche des
dänischen Mathematikers Johannes Hjelmslev zurück (*~1873, †~1950), richtig
initiiert wurde das Gebiet aber erst in den 1970er Jahren durch Anders Kock,
ebenfalls Däne.

Im Folgenden werden wir lernen, wozu infinitesimale Zahlen nützlich sind; wie
man in dem alternativen Universum arbeiten kann; und schließlich inwieweit
Resultate, die man in der neuen mathematischen Welt erzielt, auch in der
gewöhnlichen mathematischen Welt Gültigkeit haben. Ohne diesen letzten Punkt
wäre unser Unterfangen ein reines Gedankenexperiment ohne Nutzen.

Dreh- und Angelpunkt für synthetische Differentialgeometrie ist ein Axiom,
das \emph{klassisch} -- das heißt im gewöhnlichen mathematischen Universum --
falsch ist:

\begin{shaded}
\textbf{Axiom der Mikroaffinität.}
Sei~$\Delta := \{ \varepsilon \in \RR \,|\, \varepsilon^2 = 0 \}$ die
\emph{infinitesimale Monade} um~$0$. Sei~$f : \Delta \to \RR$ eine beliebige
Funktion. Dann gibt es gewisse eindeutig bestimmte reelle Zahlen~$a$ und~$b$,
sodass für alle Zahlen~$\varepsilon$ in~$\Delta$ folgende Gleichung gilt:
\[ f(\varepsilon) = a + b \varepsilon. \]
\end{shaded}

Wir werden etwas Zeit benötigen, um zunächst die Aussage dieses Axioms zu
verstehen und dann seine Konsequenzen zu überblicken.


\section{Motivation für infinitesimale Zahlen}

In Abbildung~\ref{fig:schnittverhalten} sind zwei Situationen skizziert, bei
denen sich jeweils zwei Kurven schneiden. (Gerade Linien zählen
auch als \emph{Kurven}.) Es ist offensichtlich, dass sich im ersten
Bild die beiden Geraden in genau einem Punkt schneiden. Die Schnittsituation
beim zweiten Bild ist dagegen weniger klar. In klassischer Mathematik konstatiert man,
der Schnitt bestehe ebenfalls aus nur genau einem Punkt, nämlich dem Ursprung.
In der Tat könnte man keinen weiteren Punkt benennen, der ebenfalls auf beiden
Kurven liegen würde. Anschaulich scheint es aber ja doch einen Unterschied zu
geben -- man benötigt infinitesimale Zahlen, um ihn auf direkte Art und Weise
mathematisch einzufangen.\footnote{Indirekt geht es auch in klassischer
Mathematik: Die Parabel hat bei der Stelle~$x = 0$ eine \emph{doppelte
Nullstelle}. Das bedeutet, dass nicht nur der Funktionswert dort Null ist,
sondern auch noch seine erste Ableitung.}

\begin{aufgabeShaded}{Schnittberechnung}
Die Gleichungen der beiden Kurven im ersten Bild sind
\begin{empheq}[left=\empheqlbrace\ ]{align}
  y &= 2x, \\
  y &= 0.
\end{empheq}
Die erste Gleichung gehört zur schrägen Gerade, die zweite zur~$x$-Achse (auf
der alle Punkte als~$y$-Koordinate Null haben). Die Gleichungen der Kurven im
zweiten Bild sind
\begin{empheq}[left=\empheqlbrace\ ]{align}
  y &= x^2, \\
  y &= 0.
\end{empheq}
\begin{enumerate}
\item Löse das erste Gleichungssystem, um zu beweisen: Der einzige
Schnittpunkt~$(x|y)$ hat die Koordinaten~$x = 0$ und~$y = 0$. Wieso entspricht
das Schneiden der beiden Kurven rechnerisch der Lösungsmenge des kombinierten
Gleichungssystems?
\item Löse das zweite Gleichungssystem, um zu beweisen, dass ein Punkt~$(x|y)$
genau dann im Schnittbereich des zweiten Bilds liegt, wenn seine~$x$-Koordinate
Null und seine~$y$-Koordinate eine infinitesimale Zahl ist (also~$y^2 = 0$
erfüllt).

Hierbei ist es wichtig, mit Absicht langsam zu rechnen, um nicht durch zu
schnelles Vereinfachen die Pointe vorwegzunehmen. In klassischer Mathematik
gilt die Regel "`wenn~$y^2 = 0$, dann auch~$y = 0$''; erst mit dieser Regel
vereinfacht sich das Ergebnis, dann zu demselben wie in Teilaufgabe~a).
\end{enumerate}
\end{aufgabeShaded}

Infinitesimale Zahlen sind ferner in der Physik nützlich. Dort hat man es
manchmal mit "`sehr kleinen"' (aber nicht verschwindenden) Größen wie etwa
Differenzen~$\Delta x$ zu tun. Da das Quadrat einer kleinen Zahl nochmals
kleiner und "`wirklich winzig"' ist, erlaubt man sich, in Rechnungen Quadrate
von~$\Delta x$ einfach wegzulassen. Das ist mathematisch natürlich nicht
zulässig -- trotzdem haben die Physikerinnen und Physiker mit diesem Vorgehen
offensichtlich großen Erfolg!

Mathematiker sollten die physikalischen Methoden nicht blind verurteilen,
sondern sie ernst nehmen und Möglichkeiten finden, sie mathematisch sauber und
rigoros zu verstehen. Infinitesimale Zahlen bieten eine solche Möglichkeit.sie
ernst nehmen und Möglichkeiten finden, sie mathematisch sauber und rigoros zu
verstehen. Infinitesimale Zahlen bieten eine solche Möglichkeit.

\begin{aufgabeShaded}{Quadrate großer und kleiner Zahlen}
\begin{enumerate}
\item Sei~$x$ eine reelle Zahl, die größer als~$1$ ist. Zeige: Das
Quadrat~$x^2$ ist größer als~$x$.
\item Sei~$x$ eine reelle Zahl zwischen~$0$ und~$1$. Zeige: Das Quadrat~$x^2$
ist kleiner als~$x$.

Wenn die Dezimalentwicklung von~$x$ mit~$n$ Nullen nach dem Komma beginnt, mit
etwa wie vielen Nullen beginnt dann die Dezimalentwicklung von~$x^2$?
\end{enumerate}
\end{aufgabeShaded}

Infinitesimale Zahlen sind ferner fürs Differenzieren nützlich: Sie ermöglichen
es nämlich, auf gewisse Grenzwertprozesse zu verzichten und gleichzeitig näher
an der Anschauung zu bleiben. Zur Erinnerung wollen wir die übliche Definition
der Ableitung rekapitulieren:

\begin{defn}Sei~$f : \RR \to \RR$ eine Funktion und~$x_0 \in \RR$ eine Stelle.
Falls der Grenzwert
\[ \lim_{\varepsilon \to 0} \frac{f(x_0 + \varepsilon) - f(x_0)}{\varepsilon} \]
existiert, so heißt~$f$ an der Stelle~$x_0$ \emph{differenzierbar} und die
Ableitung~$f'(x_0)$ ist per Definition dieser Grenzwert.
\end{defn}

\begin{aufgabeShaded}{Intuition zum Differentialquotienten}
Erinnere dich, inwieweit der Bruch
\[ \frac{f(x_0 + \varepsilon) - f(x_0)}{\varepsilon} \]
eine \emph{Sekantensteigung} angibt, und erkläre anhand einer Skizze, wieso der
Grenzwert für~$\varepsilon \to 0$ anschaulich die Steigung der Tangente an den
Graphen von~$f$ durch den Punkt~$(x_0|f(x_0))$ angibt.
\end{aufgabeShaded}

Um die Definition besser zu verstehen, wollen wir ohne Verwendung der bekannten
Ableitungsregeln die Ableitung der Quadratfunktion bestimmen.

\begin{prop}Die Quadratfunktion~$f : \RR \to \RR,\ x \mapsto x^2$ ist überall
differenzierbar mit Ableitung~$f'(x) = 2x$.\end{prop}
\begin{proof}In der Schule würde man in einer langen Gleichungskette die
Definition so lange vereinfachen, bis der Grenzwert unmittelbar ablesbar ist.
Dabei müsste man in jedem Schritt das~$\lim$-Symbol mitführen, würde das aber
vielleicht auch oftmals vergessen. Übersichtlicher ist folgende Schreibweise:
\[ \frac{f(x + \varepsilon) - f(x)}{\varepsilon} =
  \frac{(x + \varepsilon)^2 - x^2}{\varepsilon} =
  \frac{x^2 + 2x\varepsilon + \varepsilon^2 - x^2}{\varepsilon} =
  2x + \varepsilon \xra{\varepsilon \to 0} 2x. \qedhere \]
\end{proof}

An diesem Beweis ist an sich nichts auszusetzen. Die Rechnung selbst ist genügend
einfach, durch die Grenzwertüberlegung aber trotzdem logisch nicht ganz
elementar. Die Physiker haben eine einfachere Möglichkeit, die Aussage zu
"`beweisen"' -- eine, die ohne Grenzwerte auskommt:

\begin{proof}[Physiker-Beweis]Sei~$\varepsilon$ "`sehr klein"'. Dann gilt:
\[ \frac{f(x + \varepsilon) - f(x)}{\varepsilon} =
  \frac{(x + \varepsilon)^2 - x^2}{\varepsilon} =
  \frac{x^2 + 2x\varepsilon + \varepsilon^2 - x^2}{\varepsilon} =
  2x + \varepsilon = 2x. \qedhere \]
\end{proof}

Bemerkenswert ist die schizophrene Natur dieses Arguments:
Zum einen lässt man im letzten Schritt den Term~$\varepsilon$ einfach weg --
mit der Begründung,~$\varepsilon$ sei schließlich "`sehr klein"'. Andererseits
aber teilt man durch~$\varepsilon$ -- das ginge nur, wenn man wüsste,
dass~$\varepsilon$ positiv oder negativ, aber jedenfalls nicht Null ist.

Außerdem sollte man auch zu Beginn der Rechnung die Vereinfachung~$f(x +
\varepsilon) - f(x) = f(x) - f(x) = 0$ treffen dürfen, wenn man doch am Ende
auch einfach~$\varepsilon$ weglassen durfte. Schließlich sollten sich die
Rechengesetze nicht inmitten einer Rechnung ändern. Dann wäre das Ergebnis
insgesamt~$f'(x) = 0$ -- das ist jedoch eine unsinnige Behauptung.

Aus diesen Gründen wird das Physiker-Argument in klassischer Mathematik
nicht akzeptiert. Mit den infinitesimalen Zahlen aus synthetischer
Differentialgeometrie werden wir aber in der Lage sein, die wesentlichen Ideen des
Physiker-Arguments in einen mathematisch einwandfreien Beweis zu gießen. So
können wir die Vorteile beider Welten -- einfache Rechnungen einerseits und
mathematische Rigorosität andererseits -- vereinen.

\begin{aufgabeShaded}{Ableitung der Kubikfunktion}
Sei~$f : \RR \to \RR,\ x \mapsto x^3$ die Kubikfunktion. Verifiziere direkt
anhand der mathematischen Definition, dass~$f$ überall differenzierbar mit
Ableitung~$f'(x) = 3x^2$ ist.

\emph{Hinweis:} Den Ausdruck~$(x + \varepsilon)^3$ kann man entweder durch
mehrmaliges Ausmultiplizieren oder direkt durch Verwendung der binomischen
Formel für Kuben statt Quadrate vereinfachen. (Die Koeffizienten der
allgemeinen binomischen Formel stehen im Pascalschen Dreieck.)
\end{aufgabeShaded}

\begin{aufgabeShaded}{Beispiele für nicht differenzierbare Funktionen}
\begin{enumerate}
\item Die Betragsfunktion (siehe Abbildung~\ref{fig:nicht-diffbar}a) ist an der
Stelle~$x_0 = 0$ nicht differenzierbar. Anschaulich erkennt man das daran, dass
der Graph am Punkt~$(0|0)$ keine eindeutige Tangente besitzt.\footnotemark{}
Bestätige diese Beobachtung anhand der Definition über den
Differentialquotienten: Die Sekantensteigungen konvergieren für~$\varepsilon
\to 0$ nicht gegen einen bestimmten Wert, sondern \ldots

Wenn du möchtest, kannst du das auch noch rechnerisch nachvollziehen: Berechne
den \emph{links-} und den \emph{rechtsseitigen} Grenzwert des
Differenzenquotienten. Nutze dabei folgende Definition der Betragsfunktion:
\[ |x| := \begin{cases}\phantom{-}x, & \text{falls $x \geq 0$,} \\ -x, & \text{falls $x <
0$.}\end{cases} \]
\item Es gibt auch Funktionen, deren Graphen keine offensichtlichen
Knickpunkte haben und die trotzdem nicht überall differenzierbar sind. Erkläre
anschaulich und zeige rechnerisch, dass folgende Funktion an der Stelle~$x_0 =
0$ nicht differenzierbar ist (siehe Abbildung~\ref{fig:nicht-diffbar}b):
\[ f(x) = \begin{cases}x \sin\frac{1}{x}, & \text{falls $x \neq 0$,} \\
0, & \text{falls $x = 0$.}\end{cases} \]
(Diese Funktion ist übrigens durchaus \emph{stetig}.)
\end{enumerate}
\end{aufgabeShaded}
\footnotetext{Man kann an dieser Stelle durchaus Geraden anlegen, es gibt aber
keine besonders überzeugende Wahl einer \emph{berührenden} Gerade.
Fortgeschrittene Differentialrechnung wird mit dieser Uneindeutigkeit fertig,
durch den Begriff der \emph{Subgradienten} von konvexen Funktionen.}

\end{document}

- Mengenschreibweise (benötigt für Delta)
- Funktionen
- "Schwebezustand" von Delta: Es stimmt nicht, dass es nur die Null enthalten
  würde -- andere Elemente kann man aber trotzdem nicht angeben.
