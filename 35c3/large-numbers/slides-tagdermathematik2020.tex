\documentclass[12pt,compress,english,utf8,t]{beamer}
\usepackage[ngerman]{babel}
\usepackage{calc}
\usepackage{ragged2e,wasysym,multicol,mathtools,mathdots}
\usepackage[protrusion=true,expansion=true]{microtype}
\usepackage{tikz,ifthen}
\usetikzlibrary{calc,shapes,shapes.callouts,shapes.arrows,patterns,fit,backgrounds,decorations.pathmorphing}
\usepackage{booktabs}
\hypersetup{colorlinks=true}

\graphicspath{{images/}}

\title{Large numbers, very large numbers and very very large numbers}
\author{Ingo Blechschmidt}
\date{December 17th, 2019}

%\usetheme{Warsaw}
\useinnertheme[shadow=true]{rounded}
\useoutertheme{split}
\usecolortheme{orchid}
\usecolortheme{whale}
\setbeamerfont{block title}{size={}}

\useinnertheme{rectangles}

\usecolortheme{seahorse}
\definecolor{mypurple}{RGB}{150,0,255}
\setbeamercolor{structure}{fg=mypurple}
\definecolor{myred}{RGB}{150,0,0}
\setbeamercolor*{title}{bg=myred,fg=white}
\setbeamercolor*{titlelike}{bg=myred,fg=white}

\usefonttheme{serif}
\usepackage[T1]{fontenc}
\usepackage{libertine}

\renewcommand{\_}{\mathpunct{.}\,}
\newcommand{\BB}{\mathbb{B}}
\newcommand{\M}{\mathcal{M}}
\newcommand{\R}{\mathrm{R}}
\newcommand{\NN}{\mathbb{N}}
\newcommand{\RR}{\mathbb{R}}

\setbeamertemplate{navigation symbols}{}

\setbeamertemplate{title page}[default][colsep=-1bp,rounded=false,shadow=false]
\setbeamertemplate{frametitle}[default][colsep=-2bp,rounded=false,shadow=false,center]

\newcommand{\hil}[1]{{\usebeamercolor[fg]{item}{\textbf{#1}}}}
\setbeamertemplate{frametitle}{%
  \vskip1em%
  \leavevmode%
  \begin{beamercolorbox}[dp=1ex,center]{}%
      \usebeamercolor[fg]{item}{\textbf{\textsf{\Large \insertframetitle}}}
  \end{beamercolorbox}%
}

\setbeamertemplate{footline}{%
  \leavevmode%
  \hfill%
  \begin{beamercolorbox}[ht=2.25ex,dp=1ex,right]{}%
    \usebeamerfont{date in head/foot}
    \insertframenumber\,/\,\inserttotalframenumber\hspace*{1ex}
  \end{beamercolorbox}%
  \vskip0pt%
}

\newcommand{\backupstart}{
  \newcounter{framenumberpreappendix}
  \setcounter{framenumberpreappendix}{\value{framenumber}}
}
\newcommand{\backupend}{
  \addtocounter{framenumberpreappendix}{-\value{framenumber}}
  \addtocounter{framenumber}{\value{framenumberpreappendix}}
}

\setbeameroption{show notes}
\setbeamertemplate{note page}[plain]

\input{images/primes}

\setbeamertemplate{headline}{
  \begin{beamercolorbox}[wd=\paperwidth,ht=2.25ex]{}%
  \end{beamercolorbox}
}

\addtocounter{framenumber}{-1}

\newcommand{\imgslideHeight}[1]{{\usebackgroundtemplate{\parbox[c][\paperheight][c]{\paperwidth}{\centering\includegraphics[height=\paperheight]{#1}}}\begin{frame}[plain]\end{frame}}}
\newcommand{\imgslideWidth}[1]{{\usebackgroundtemplate{\parbox[c][\paperheight][c]{\paperwidth}{\centering\includegraphics[width=\paperwidth]{#1}}}\begin{frame}[plain]\end{frame}}}

\begin{document}

{\usebackgroundtemplate{\begin{minipage}{\paperwidth}\vspace*{0.3cm}\centering\scriptsize\sieve{25}{2}\\\vspace*{3.95cm}\includegraphics[width=\paperwidth]{sun3}\end{minipage}}
\begin{frame}[c]
  \centering

  \bigskip
  \bigskip
  \bigskip

  \hil{Herkulas Kampf gegen die Hydra:}

  \hil{Große Zahlen,}
  
  \large
  \hil{sehr große Zahlen,}
  
  \Large
  \hil{sehr sehr große Zahlen und}

  \Large
  \scalebox{1.2}{\hil{mehr als unendlich große Zahlen}}

  \bigskip
  \scriptsize
  \textit{-- eine Einladung in fortgeschrittene Googologie --}
  \bigskip
  \bigskip
  \bigskip
  \bigskip
  \medskip

  \textcolor{black}{
    \textbf{Tag der Mathematik am 7. März 2020} \\
    \emph{Fragen sind jederzeit willkommen! Bitte nicht bis zum Ende aufsparen.}
  }
  \bigskip
  \bigskip

  \tiny
  \textcolor{white}{
    Ingo Blechschmidt \\
    Lehrstuhl für Nichtlineare Analysis
  }

  \par
\end{frame}}


\section{Large numbers}

\tikzstyle{card}   = [draw=mypurple, very thick, rectangle, rounded corners, inner sep=5pt, inner ysep=10pt]
\tikzstyle{author} = [fill=mypurple, text=white]
\tikzstyle{descr}  = []

\newcommand{\card}[2]{
  \begin{tikzpicture}
    \node[descr]  (descr)  {#2};
    \node[card, tape] [fit = (descr)] (card) {};
    \node[author] at (card.north) (author) {#1};
  \end{tikzpicture}
}

{\usebackgroundtemplate{\begin{minipage}{\paperwidth}\vspace*{6.5cm}\includegraphics[width=\paperwidth]{sandstrand}\end{minipage}}
\begin{frame}
  \centering

  \Huge \hil{Teil 0}

  \bigskip
  \Large\textbf{Große Zahlen}
  \par
  \bigskip
  \bigskip

  \normalsize

  \hil{300\,000} Augsburger*innen
  \medskip

  \hil{$\boldsymbol{10^{19} = 1\!\underbrace{0\,000\,000\,000\,000\,000\,000}_{\text{$19$ Nullen}}}$}
  Sandkörner auf der Erde
  \medskip

  \hil{$\boldsymbol{10^{80} = 1\!\underbrace{000\ldots000}_{\text{$80$ Nullen}}}$}
  Elementarteilchen im Universum

  % ACTION: video of mass of black hole https://www.youtube.com/watch?v=aIg8gAbqqZc?t=176
  % ACTION: 52!
\end{frame}}

\imgslideHeight{milky-way}
\imgslideHeight{ocean}


\section{Sehr große Zahlen}

\begin{frame}
  \centering

  \Huge \hil{Teil I}

  \bigskip
  \Large\textbf{Sehr große Zahlen}
  \par
  \bigskip

  \normalsize

  \only<1-10>{\[\begin{aligned}
    2 \cdot 4 &= 2 + 2 + 2 + 2 = 8 \\
    2^4 &= 2 \cdot 2 \cdot 2 \cdot 2 = 16 \\
    \pause
    2 \uparrow\uparrow 4 &= 2^{2^{2^2}} = 2^{2^4} = 2^{16} = 65\,536 \\
    \pause
    2 \uparrow\uparrow\uparrow 4 &= 2 \uparrow\uparrow (2 \uparrow\uparrow (2
    \uparrow\uparrow 2)) \pause =
    2 \uparrow\uparrow (2 \uparrow\uparrow 4) \pause =
    2 \uparrow\uparrow 65\,536 \\ \pause &
    \left.\kern-3\nulldelimiterspace
    \begin{array}{@{}l@{}}{}= 2^{2^{\iddots^2}}\end{array} \right\rbrace \text{\scriptsize 65\,536 viele Zweien} \\
    \pause
    2 \uparrow\uparrow\uparrow\uparrow 4 &= 2 \uparrow\uparrow\uparrow (2
    \uparrow\uparrow\uparrow (2 \uparrow\uparrow\uparrow 2)) \pause =
    2 \uparrow\uparrow\uparrow (2 \uparrow\uparrow\uparrow 4) \pause \\
    &= 2 \uparrow\uparrow\uparrow 2^{2^{\iddots^2}} \pause =
    \underbrace{2 \uparrow\uparrow (2 \uparrow\uparrow (2 \uparrow\uparrow (\cdots
    \uparrow\uparrow 2)))}_{\text{$2^{2^{\iddots^2}}$ viele Zweien}}
  \end{aligned}\]}

  \only<11->{\[
    \left.
      \begin{matrix}
        \text{\hil{Grahams Zahl}} &=&3\underbrace{\uparrow \cdots \cdots \cdots \cdots \cdots \uparrow}3 \\
          & &3\underbrace{\uparrow \cdots \cdots \cdots \cdots \uparrow}3 \\[-0.4em]
          & & \underbrace{\qquad \quad \vdots \qquad \quad} \\
          & &3\underbrace{\uparrow \cdots \cdots \uparrow}3 \\
          & &3\uparrow \uparrow \uparrow \uparrow3
      \end{matrix}
    \right \} \text{\small 64 Ebenen}
  \]}

  \only<12->{
    \vspace*{-2.5cm}
    \includegraphics{monophilic-coloring}\hspace*{6cm}
  }
\end{frame}

\begin{frame}[plain]
  \Huge\centering\vfill
  \scalebox{1.5}{$\sqrt{2}^{\sqrt{2}^{\sqrt{2}^{\sqrt{2}^{\iddots}}}} = 2$}
  \pause
  \bigskip
  \bigskip

  \large
  $\sqrt{2 + \sqrt{2 + \sqrt{2 + ...}}} = 2$
  \bigskip

  $\frac{2}{\pi} = \sqrt{\tfrac{1}{2}} \cdot
    \sqrt{\tfrac{1}{2} + \tfrac{1}{2} \sqrt{\tfrac{1}{2}}} \cdot
    \sqrt{\tfrac{1}{2} + \tfrac{1}{2} \sqrt{\tfrac{1}{2} + \tfrac{1}{2}
    \sqrt{\tfrac{1}{2}}}} \cdot \ldots$
  \vfill
\end{frame}


\section{Sehr sehr große Zahlen}

\begin{frame}
  \centering

  \Huge \hil{Teil II}

  \bigskip
  \Large\textbf{Sehr sehr große Zahlen}
  \par
  \bigskip

  \normalsize

  \includegraphics[width=0.8\textwidth]{tree3}
  \bigskip

  Jeder Wald stirbt schlussendlich, bei einer Maximalzahl
  von~\hil{$\boldsymbol{\mathrm{TREE}(3)}$} Bäumen.
\end{frame}


\section{Sehr sehr sehr große Zahlen}

\begin{frame}
  \centering

  \Huge \hil{Teil III}

  \bigskip
  \Large\textbf{Sehr sehr sehr große Zahlen}
  \par
  \bigskip

  \normalsize

  \includegraphics[width=0.3\textwidth]{turing-machine}

  \begin{itemize}\justifying
    \item \hil{$\boldsymbol{\mathrm{BB}(n)}$} ist die größte Zahl, die ein
    terminierendes Computerprogramm bestehend aus~$n$ Bytes berechnen kann.
    \pause
    % ACTION: Values on Wikipedia
    \item Die Busy-Beaver-Funktion ist \hil{unberechenbar}\only<2>{.}
    \visible<3->{und \hil{dominiert} jede berechenbare Funktion.}
    \pause
    \pause
    \item Keine Vermutung über den Wert von~$\mathrm{BB}(1919)$ ist
    mathematisch beweisbar, noch nicht einmal~``$\mathrm{BB}(1919) =
    \heartsuit$'' wobei~$\heartsuit$ der wahre Wert von~$\mathrm{BB}(1919)$ ist.
  \end{itemize}
\end{frame}


%\section{Honorable mentions}
%
%\begin{frame}
%  \centering
%
%  \Huge \hil{Part IV}
%
%  \bigskip
%  \Large\textbf{Honorable mentions}
%  \par
%  \bigskip
%  \normalsize
%
%  \begin{columns}
%    \begin{column}{0.4\textwidth}
%      \card{}{\begin{minipage}{3.3cm}\raggedright The submission of
%      \texttt{\$THAT\textunderscore PERSON}, with the following modification: \ldots\end{minipage}}
%    \end{column}
%
%    \begin{column}{0.4\textwidth}
%      \card{}{\begin{minipage}{4cm}The largest submission of this contest, plus
%      one.\end{minipage}}
%    \end{column}
%
%    \begin{column}{0.4\textwidth}
%      \card{}{\begin{minipage}{4cm}The largest of all \emph{other} submissions
%      of this contest, plus one.\end{minipage}}
%    \end{column}
%  \end{columns}
%\end{frame}


\section{Sehr sehr sehr sehr große Zahlen}

{\usebackgroundtemplate{\begin{minipage}{\paperwidth}\vspace*{0.3cm}\centering\scriptsize\sieve{25}{2}\\\vspace*{3.95cm}\includegraphics[width=\paperwidth]{sun3}\end{minipage}}
\begin{frame}
  \centering

  \Huge \hil{\phantom{Teil IV}}

  \bigskip
  \Large\textbf{Sehr sehr sehr sehr große Zahlen}
  \par
  \bigskip
  \normalsize

  \begin{itemize}\justifying
    \item \hil{$\boldsymbol{\mathrm{Rayo}(n)}$} ist die größte Zahl, die man
    mit höchstens~$n$ Zeichen mathematisch präzise beschreiben kann.
    \item Die Rayo-Funktion \hil{dominiert} \emph{jede}
    mathematisch präzise definierbare Funktion.
  \end{itemize}

  \vfill\hfill\includegraphics[width=2em]{trollface}
\end{frame}}


\section{Unendlich große Zahlen}

{\usebackgroundtemplate{\begin{minipage}{\paperwidth}\vspace*{0.3cm}\centering\scriptsize\sieve{25}{2}\\\vspace*{3.95cm}\includegraphics[width=\paperwidth]{buergerbuero-haunstetten}\end{minipage}}
\begin{frame}
  \centering

  \Huge \hil{\phantom{Teil V}}

  \bigskip
  \Large\textbf{Unendlich große Zahlen}
  \par
  \bigskip

  \large\hil{Ordinalzahlen}
  \par

  messen Anordnung
\end{frame}}

{\usebackgroundtemplate{\parbox[c][\paperheight][c]{\paperwidth}{\vspace*{0.2cm}\hspace*{-0.2cm}\includegraphics[height=1.07\paperheight]{fff-2019-12-20}}}\begin{frame}[plain]\end{frame}}
\end{document}


BEACHTEN:
* Black hole
* Wikipedia-Werte BB
* Nutzen
* Ziffern Graham

NÄCHSTES MAL BESSER MACHEN:
* Beim BB- und Rayo-Kapitel mehr Drive haben.
* Krassheit von Rayo besser zur Geltung bringen.
* Nutzen vorbringen!
* Ruhiger formale Systeme erklären?
* "Calbrating the finite by using the infinite" (fast-growing hierarchy)
