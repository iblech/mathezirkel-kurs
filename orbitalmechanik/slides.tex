\documentclass[12pt,compress,ngerman,utf8,t]{beamer}
\usepackage[ngerman]{babel}
\usepackage{calc}
\usepackage{ragged2e,wasysym,multicol,mathtools}
\usepackage[protrusion=true,expansion=true]{microtype}
\usepackage{booktabs}
\hypersetup{colorlinks=true}

\graphicspath{{images/}}

\title[Orbital mechanics and low-energy transfers]{\large How space travel is revolutionized with this one weird trick from chaos theory}
\author[Ingo Blechschmidt]{\textcolor{white}{Ingo Blechschmidt \\ thanking Sven Prüfer}}
\date[2016-12-30]{\vspace*{-5em}\ \\\textcolor{white}{\scriptsize Institut für Mathematik \\ Universität Augsburg \\ December 30th, 2016 \\}}

%\usetheme{Warsaw}
\useinnertheme[shadow=true]{rounded}
\useoutertheme{split}
\usecolortheme{orchid}
\usecolortheme{whale}
\setbeamerfont{block title}{size={}}

\useinnertheme{rectangles}

\usecolortheme{seahorse}
\definecolor{mypurple}{RGB}{150,0,255}
\setbeamercolor{structure}{fg=mypurple}
\definecolor{myred}{RGB}{150,0,0}
\setbeamercolor*{title}{bg=myred,fg=white}
\setbeamercolor*{titlelike}{bg=myred,fg=white}

\usefonttheme{serif}
\usepackage[T1]{fontenc}
\usepackage{libertine}

\renewcommand{\_}{\mathpunct{.}\,}
\newcommand{\BB}{\mathbb{B}}
\newcommand{\M}{\mathcal{M}}
\newcommand{\R}{\mathrm{R}}
\newcommand{\NN}{\mathbb{N}}
\newcommand{\RR}{\mathbb{R}}

\newcommand{\slogan}[1]{%
  \begin{center}%
    \setlength{\fboxrule}{2pt}%
    \setlength{\fboxsep}{8pt}%
    {\usebeamercolor[fg]{item}\fbox{\usebeamercolor[fg]{normal text}\parbox{0.91\textwidth}{#1}}}%
  \end{center}%
}

\setbeamertemplate{navigation symbols}{}

\setbeamertemplate{title page}[default][colsep=-1bp,rounded=false,shadow=false]
\setbeamertemplate{frametitle}[default][colsep=-2bp,rounded=false,shadow=false,center]

\newcommand{\hil}[1]{{\usebeamercolor[fg]{item}{\textbf{#1}}}}
\setbeamertemplate{frametitle}{%
  \vskip1em%
  \leavevmode%
  \begin{beamercolorbox}[dp=1ex,center]{}%
      \usebeamercolor[fg]{item}{\textbf{\textsf{\Large \insertframetitle}}}
  \end{beamercolorbox}%
}

\setbeamertemplate{footline}{%
  \leavevmode%
  \hfill%
  \begin{beamercolorbox}[ht=2.25ex,dp=1ex,right]{}%
    \usebeamerfont{date in head/foot}
    \insertframenumber\,/\,\inserttotalframenumber\hspace*{1ex}
  \end{beamercolorbox}%
  \vskip0pt%
}

\newcommand{\backupstart}{
  \newcounter{framenumberpreappendix}
  \setcounter{framenumberpreappendix}{\value{framenumber}}
}
\newcommand{\backupend}{
  \addtocounter{framenumberpreappendix}{-\value{framenumber}}
  \addtocounter{framenumber}{\value{framenumberpreappendix}}
}

\setbeameroption{show notes}
\setbeamertemplate{note page}[plain]

\begin{document}

% http://www.ufointernationalproject.com/wp-content/uploads/2015/11/a23.jpg
{\usebackgroundtemplate{\includegraphics[height=\paperheight]{images/interstellar}}
\frame{\vspace*{9em}\titlepage}}
\frame{\tableofcontents}


\section[Crash course]{A crash course on orbital mechanics}

{\usebackgroundtemplate{\begin{minipage}{\paperwidth}\vspace*{6.5cm}\includegraphics[width=\paperwidth]{images/orbit-wide}\end{minipage}}
\begin{frame}
  \centering
  \bigskip\bigskip

  \Huge \hil{Part I}

  \bigskip
  \Large\textbf{A crash course on orbital mechanics}
  \par
\end{frame}}

{\usebackgroundtemplate{\begin{minipage}{\paperwidth}\hfill\includegraphics[height=\paperheight]{images/orbit-tall}\end{minipage}}
\begin{frame}{Basic facts}
  \begin{itemize}
    \item Getting to space is easy. \\ The hard part is staying there.
    % Mention: no friction.
  \end{itemize}
\end{frame}}


\subsection{Basic facts}

\begin{frame}{Basic facts}
  \begin{itemize}
    \item<1-> Getting to space is easy. \\ The hard part is staying there.
    \item<3-> Velocity is very important.
    \item<4-> In the \hil{one-body problem}, there are only three kinds of
    orbits: elliptic, parabolic, and hyperbolic.
    \item<5-> Have your models straight: Earth is \ldots
    \begin{enumerate}
      \item a perfect ball?
      \item has atmosphere?
      \item rotating?
    \end{enumerate}
    % instantaneous impulses?
  \end{itemize}

  \only<1>{
    \centering
    % https://upload.wikimedia.org/wikipedia/commons/thumb/7/73/Newton_Cannon.svg/240px-Newton_Cannon.svg.png
    \vspace*{-7em}
    \includegraphics[width=0.4\textwidth]{images/newton-cannon}

    $F_{\text{centripetal}} = F_{\text{gravitation}}
    \leadsto v_1 = \sqrt{GM/r}$

    $E_{\text{kinetic}} = E_{\text{gravitation}}
    \leadsto v_2 = \sqrt{2} \, v_1$
    \par
  }

  \only<2>{
    \centering
    \vspace*{-4em}
    \begin{tabular}{ll}
      \toprule
      body & second escape velocity \\\midrule
      Earth & $11.2 \,\mathrm{km}/\mathrm{s} \approx 40\,000 \,\mathrm{km}/\mathrm{h}$ \\
      Moon  & $2.4  \,\mathrm{km}/\mathrm{s}$ \\
      Sun   & $618  \,\mathrm{km}/\mathrm{s}$ \\
      Milky Way & $\approx 550 \,\mathrm{km}/\mathrm{s}$ \\
      \bottomrule
    \end{tabular}
    \par
  }
\end{frame}


\subsection[Changing orbits]{Changing orbits}

\begin{frame}{Changing orbits}
  \begin{center}
    \hil{``Live demo''}
  \end{center}

  \begin{itemize}
    \item Changing the eccentricity
    % Accelerate in tangential direction

    \item Changing the radius
    % Hohmann transfer

    \item Changing the phase
    % Speed away from the central body.
    % Necessary Δv depends on eccentricity.
    % If eccentricity is zero, need to Δv.

    \item Changing inclination
    % Easy, just accelerate normal to the plane.
  \end{itemize}
\end{frame}


\subsection[Rocket equation]{The tyranny of the rocket equation}

\begin{frame}{The tyranny of the rocket equation}
  \centering
  % http://tsiolkovsky.org/wp-content/uploads/2015/02/icon_en.png
  \includegraphics[width=0.4\textwidth]{images/tsiolkovsky}
  \bigskip

  \scalebox{2}{$m_\text{total} = m_\text{payload} \cdot e^{\Delta v / v_\text{eff.\@ exhaust}}$}
  \par
\end{frame}

% Weltall ist nah.
% Extrem wichtig: (Ort,Geschwindigkeit) des Ziels
% Erste Fluchtgeschwindigkeit
% Filmausschnitt Bruce Willis
% Filmausschnitt Gravity?
% Arten von Trajektorien (stets Ellipsen, Parabeln oder Hyperbeln)
% Tyrannei der Raketengleichung
% Orbitwechsel
% Gravity assist

\section[Chaos theory]{One weird trick from chaos theory}

{\usebackgroundtemplate{\includegraphics[height=\paperheight]{images/Interplanetary_Superhighway}}
\begin{frame}
  \centering
  \bigskip\bigskip

  \Huge \hil{Part II}

  \bigskip
  \Large\textcolor{white}{\textbf{One weird trick from chaos theory}}
  \par
\end{frame}}

% Probleme mit konventionellem Ansatz
% Ziel: ballistisches Einfangen
% Lagrange-Punkte
% dazu mitrotierendes Bezugssystem, Zentrifugalkraft, Corioliskraft
% WSM
% Wie man sie findet
% Hiten
% In der Natur

\end{document}
