\documentclass[12pt,compress,ngerman,utf8,t]{beamer}
\usepackage{etex}
\usepackage[ngerman]{babel}
\usepackage{calc,dashrule,tabto,tikz}
\usetikzlibrary{arrows}
\usepackage[protrusion=true,expansion=true]{microtype}
\usepackage[normalem]{ulem}

\graphicspath{{images/}}

\title[Unendlich große Zahlen]{\bf Die wundersame Welt der \\ unendlich großen Zahlen
\\[0.5em] \normalsize Glaube in der Mathematik?}
\author[Ingo Blechschmidt]{\textcolor{white}{Königsbrunner Campus \\ 10. Mai 2017}}
\date[2017-05-10]{\vspace*{6em}\ \\\textcolor{white}{Ingo Blechschmidt \\ \scriptsize
Lehrstuhl für Algebra und Zahlentheorie an der Universität Augsburg \\}}

\useinnertheme[shadow=true]{rounded}
\useoutertheme{split}
\usecolortheme{orchid}
\usecolortheme{whale}
\setbeamerfont{block title}{size={}}

\useinnertheme{rectangles}

\usecolortheme{seahorse}
\definecolor{mypurple}{RGB}{150,0,255}
\setbeamercolor{structure}{fg=mypurple}
\definecolor{myred}{RGB}{150,0,0}
\setbeamercolor*{title}{fg=white}
\setbeamercolor*{titlelike}{bg=myred,fg=white}

\usefonttheme{serif}
\usepackage[T1]{fontenc}
\usepackage{libertine}

\setbeamertemplate{navigation symbols}{}

\setbeamertemplate{title page}[default][colsep=-1bp,rounded=false,shadow=false]
\setbeamertemplate{frametitle}[default][colsep=-2bp,rounded=false,shadow=false,center]

\newcommand{\hil}[1]{{\usebeamercolor[fg]{item}{\textbf{#1}}}}
\setbeamertemplate{frametitle}{%
  \vskip1em%
  \leavevmode%
  \begin{beamercolorbox}[dp=1ex,center]{}%
      \usebeamercolor[fg]{item}{\textbf{\textsf{\Large \insertframetitle}}}
  \end{beamercolorbox}%
}

\setbeamertemplate{footline}{%
  \leavevmode%
  \hfill%
  \begin{beamercolorbox}[ht=2.25ex,dp=1ex,right]{}%
    \usebeamerfont{date in head/foot}
    \insertframenumber\,/\,\inserttotalframenumber\hspace*{1ex}
  \end{beamercolorbox}%
  \vskip0pt%
}

\begin{document}

{\usebackgroundtemplate{\includegraphics[width=\paperwidth]{infinity-space}}
\frame{\vspace*{-1em}\titlepage}}

\begin{frame}{Gliederung}
  \tableofcontents

  \centering
  Fragen sind während des gesamten Vortrags willkommen. \\
  Bitte keinesfalls bis zum Ende aufsparen.
  \par
\end{frame}

\section{Ordinalzahlen}
\only<2>{\includegraphics[width=\paperwidth]{buergerbuero-haunstetten}}

% https://commons.wikimedia.org/wiki/File:RathausKoenigsbrunn.jpg
{\usebackgroundtemplate{\begin{minipage}{\paperwidth}\vspace*{1cm}\includegraphics[width=\paperwidth]{rathaus-koenigsbrunn}\end{minipage}}
\begin{frame}
  \centering
  \bigskip

  \Huge \hil{Teil I}

  \bigskip
  \Large\textbf{Ordinalzahlen}
  \par

  messen Anordnung
  \par
\end{frame}}

{\usebackgroundtemplate{\begin{minipage}{\paperwidth}\vspace*{5cm}\includegraphics[width=\paperwidth]{buergerbuero-haunstetten}\end{minipage}}
\begin{frame}
  \centering
  \bigskip

  \Huge \hil{Teil I}

  \bigskip
  \Large\textbf{Ordinalzahlen}
  \par

  messen Anordnung
  \par
\end{frame}}


\section{Kardinalzahlen}

\begin{frame}
  \centering
  \bigskip

  \Huge \hil{Teil II}

  \bigskip
  \Large\textbf{Kardinalzahlen}
  \par

  messen Anzahl
  \par
  \bigskip
  \bigskip

  \only<1>{
    \begin{columns}[t]
      \begin{column}{0.2\textwidth}\end{column}
      \begin{column}{0.3\textwidth}
        \centering
        \includegraphics[height=0.33\textheight]{hilbert} \\
        {\scriptsize David Hilbert \\ * 1862 \\ † 1943\par}
      \end{column}
      \begin{column}{0.3\textwidth}
        \centering
        \includegraphics[height=0.33\textheight]{noether} \\
        {\scriptsize Emmy Noether \\ * 1882 \\ † 1935\par}
      \end{column}
      \begin{column}{0.2\textwidth}\end{column}
    \end{columns}
  }

  \raggedright
  \only<2->{\begin{columns}[b]
    \begin{column}{0.05\textwidth}\end{column}
    \begin{column}{0.65\textwidth}
      \visible<3->{Es gibt \hil{$\boldsymbol{\aleph_0}$} viele natürliche Zahlen: 1, 2, 3, \ldots}
      \bigskip

      \visible<4->{$\aleph_0 + 1 = \aleph_0$}
      \quad
      \visible<5->{$\aleph_0 \cdot \aleph_0 = \aleph_0$}
      \bigskip

      \visible<6->{Es gibt \hil{$\boldsymbol{\mathfrak{c}}$} viele reelle Zahlen:
      $\sqrt{2}$, $\pi$, $e$, \ldots}
    \end{column}
    \begin{column}{0.45\textwidth}
      \includegraphics[width=5.5cm]{hilberts-hotel}
      \vspace*{-0.4cm}
    \end{column}
  \end{columns}}
\end{frame}


\section{Erkenntnistheorie}

\begin{frame}
  \centering
  \bigskip

  \Huge \hil{Teil III}

  \bigskip
  \Large\textbf{Erkenntnistheorie}
  \par
  \normalsize
  \bigskip
  \bigskip
  \bigskip
  \bigskip

  \begin{columns}[t]
    \begin{column}{0.32\textwidth}
      \centering\includegraphics[width=0.7\textwidth]{p-adic-numbers}
      \medskip

      "`Es gibt unendlich viele Primzahlen."'
    \end{column}
    \begin{column}{0.32\textwidth}
      \centering\includegraphics[width=0.9\textwidth]{platonic-solids}
      \medskip

      "`Es gibt nur fünf platonische Körper."'
    \end{column}
    \begin{column}{0.37\textwidth}
      \centering\includegraphics[width=0.7\textwidth]{hafez-tomb}
      \medskip

      "`Der goldene Schnitt ist die irrationalste Zahl."'
    \end{column}
  \end{columns}
\end{frame}

\begin{frame}{Die Kontinuumshypothese}
  \begin{columns}[t]
    \begin{column}{0.34\textwidth}
      \centering\includegraphics[height=0.5\textheight]{georg-cantor} \\
      {\scriptsize Georg Cantor (* 1845, † 1918)\par}
      \bigskip

      Gibt es eine Zwischenstufe zwischen $\aleph_0$ und $\mathfrak{c}$?
    \end{column}
    \pause

    \begin{column}{0.34\textwidth}
      \centering\includegraphics[height=0.5\textheight]{kurt-goedel} \\
      {\scriptsize Kurt Gödel (* 1906, † 1978)\par}
      \bigskip

      Es gibt keinen Beweis, dass es eine Zwischenstufe gibt.
    \end{column}
    \pause

    \begin{column}{0.34\textwidth}
      \centering\includegraphics[height=0.5\textheight]{paul-cohen} \\
      {\scriptsize Paul Cohen (* 1934, † 2007)\par}
      \bigskip

      Es gibt keinen Beweis, dass es keine Zwischenstufe gibt.
    \end{column}
  \end{columns}
\end{frame}


\section*{}

\begin{frame}
  \centering
  \bigskip

  \Huge \hil{Abschluss}
  \bigskip

  \normalsize
  \raggedright

  \begin{itemize}
    \item Ordinalzahlen messen Anordnung.
    $\omega + 1 > \omega$

    \item Kardinalzahlen messen Anzahl.
    $\aleph_0 + 1 = \aleph_0$

    \item Es gibt mathematische Aussagen, die bewiesenermaßen dauerhaft
    unkennbar sind.
  \end{itemize}
  \pause
  \bigskip

  \small
  Weitere Termine in dieser Reihe:
  \begin{itemize}
    \item 11.10.: Das Verhältnis von Literatur und (Natur-)Wissenschaft \\[-1em]
    \item 15.11.: Couchsurfing: Neue Freunde oder New Economy?
  \end{itemize}
\end{frame}

{\usebackgroundtemplate{\begin{minipage}{\textwidth}\vspace*{-0.5cm}\includegraphics[width=\paperwidth]{mathecamp-gruppenfoto}\end{minipage}}
\begin{frame}[plain]
  \vspace{\textheight}\vspace{0.1em}\centering
  \hil{Mathecamp vom 19. bis 27. August in Violau}
  \par
\end{frame}}
\addtocounter{framenumber}{-2}

\end{document}

% Grab des Hafez in Shiraz (Iran), Aufnahme von Pentocelo
% Kontinuumshypothese: Aussage, dass es keine Zwischenstufe gibt.
% Cantors Begründung der Mengenlehre: 1874 bis 1897
% Cantors Vermutung: 1878
% Gödels Beweis: 1938
% Cohens Beweis: 1963
% Protagoras: möglicherweise * 490 v. Chr., † 411 v. Chr.
