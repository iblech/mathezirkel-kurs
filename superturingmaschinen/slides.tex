\documentclass[12pt,compress,ngerman,utf8,t]{beamer}
\usepackage[ngerman]{babel}
\usepackage{ragged2e,wasysym,multicol}
\usepackage[protrusion=true,expansion=true]{microtype}

\hypersetup{colorlinks=true}

\title[Superturingmaschinen]{Superturingmaschinen}
\author[Ingo Blechschmidt]{\textcolor{white}{Ingo Blechschmidt}}
\date[2016-10-04]{\vspace*{-4em}\ \\\textcolor{white}{\scriptsize Augsburger Curry-Club \\ 4.
Oktober 2016\par}}

\usetheme{Warsaw}

\useinnertheme{rectangles}

\usecolortheme{seahorse}
\definecolor{mypurple}{RGB}{150,0,255}
\setbeamercolor{structure}{fg=mypurple}
\definecolor{myred}{RGB}{150,0,0}
\setbeamercolor*{title}{bg=myred,fg=white}
\setbeamercolor*{titlelike}{bg=myred,fg=white}

\usefonttheme{serif}
\usepackage[T1]{fontenc}
\usepackage{libertine}

\newcommand{\NN}{\mathbb{N}}

\newcommand{\slogan}[1]{%
  \begin{center}%
    \setlength{\fboxrule}{2pt}%
    \setlength{\fboxsep}{8pt}%
    {\usebeamercolor[fg]{item}\fbox{\usebeamercolor[fg]{normal text}\parbox{0.91\textwidth}{#1}}}%
  \end{center}%
}

\setbeamertemplate{navigation symbols}{}
\setbeamertemplate{headline}{}

\setbeamertemplate{title page}[default][colsep=-1bp,rounded=false,shadow=false]
\setbeamertemplate{frametitle}[default][colsep=-2bp,rounded=false,shadow=false,center]

\newcommand*\oldmacro{}%
\let\oldmacro\insertshorttitle%
\renewcommand*\insertshorttitle{%
  \oldmacro\hfill\insertframenumber\,/\,\inserttotalframenumber\hfill}

\newcommand{\hil}[1]{{\usebeamercolor[fg]{item}{\textbf{#1}}}}
\setbeamertemplate{frametitle}{%
  \vskip1em%
  \leavevmode%
  \begin{beamercolorbox}[dp=1ex,center]{}%
      \usebeamercolor[fg]{item}{\textbf{\textsf{\Large \insertframetitle}}}
  \end{beamercolorbox}%
}

\setbeamertemplate{footline}{%
  \leavevmode%
  \hfill%
  \begin{beamercolorbox}[ht=2.25ex,dp=1ex,right]{}%
    \usebeamerfont{date in head/foot}
    \insertframenumber\,/\,\inserttotalframenumber\hspace*{1ex}
  \end{beamercolorbox}%
  \vskip0pt%
}

\newcommand{\backupstart}{
  \newcounter{framenumberpreappendix}
  \setcounter{framenumberpreappendix}{\value{framenumber}}
}
\newcommand{\backupend}{
  \addtocounter{framenumberpreappendix}{-\value{framenumber}}
  \addtocounter{framenumber}{\value{framenumberpreappendix}}
}

\setbeameroption{show notes}
\setbeamertemplate{note page}[plain]

\begin{document}

% http://www.ufointernationalproject.com/wp-content/uploads/2015/11/a23.jpg
{\usebackgroundtemplate{\includegraphics[height=\paperheight]{images/interstellar}}
\frame{\tableofcontents}}

\section{Erinnerungen}

\subsection{Gewöhnliche Turingmaschinen}
\begin{frame}{Ein Hoch auf Turingmaschinen}
  \begin{center}
    \includegraphics[width=0.6\textwidth]{images/turing-machine}
  \end{center}

  \begin{multicols}{2}
    \begin{enumerate}
      \item Schlichtheit
      \item Mechanischer Bezug
      \item Robustheit des Konzepts
      \item Äquivalenz zu anderen Modellen
      \item Querverbindungen
    \end{enumerate}
  \end{multicols}
\end{frame}
% mündlich: Es gibt Turingmaschinen mit 2 Zuständen und 4 Symbolen
% sowie mit 3 Zuständen und 3 Symbolen, deren Halteverhalten unbekannt ist.

\begin{frame}{Fun Facts zu Turingmaschinen}
  \begin{enumerate}
    \item Schon kleine Turingmaschinen sind diffizil.
    \item Es gibt Turingmaschinen, deren Halteverhalten unabhängig von
    Standard-Axiomen der Mathematik ist.
    % Zum Beispiel: die TM, die nach einem Widerspruch in ZFC sucht.
    % Wenn ZFC konsistent ist, dann hält diese nicht.
    % Dieser Umstand ist aber nicht in ZFC beweisbar (Gödel II).
    \pause

    \item Alle sinnvollen Modelle für Berechenbarkeit stimmen für
    Funktionen~$\NN \to \NN$ überein.
    % Aber nicht für Funktionen höherer Ordnung!
    \pause

    \item Eine Menge ist genau dann rekursiv aufzählbar, wenn
    sie durch eine~$\Sigma_1$-Aussage definierbar ist:
    \[ \{ n \in \NN \,|\, \text{es gibt $m \in \NN$ mit $\heartsuit$} \}, \]
    \pause
    und wenn sie diophantisch ist:
    \[ \{ n \in \NN \,|\, \text{die Gl. $f(n,x_1,\ldots,x_m) = 0$
    besitzt eine Lösung} \}, \]
    wobei $f$ ein Polynom mit ganzzahligen Koeffizienten ist.
  \end{enumerate}
\end{frame}


\subsection{Ordinalzahlen}

\begin{frame}{Ordinalzahlen messen Anordnung}
\end{frame}


\section{Erste Schritte mit Superturingmaschinen}

\subsection{Definition}

\begin{frame}{Was sind Superturingmaschinen?}
\end{frame}


\end{document}

Ein Hoch auf Turingmaschinen
* Einfachheit
* Maschinelle Umsetzung klar
* Robustheit des Konzepts
* Äquivalenz zu anderen Modellen (aber nur für N --> N)
* Verbindungen zur Logik
* (aber: "TM sind nicht alles", siehe zum Beispiel R-Maschinen oder QTM)

Crashkurs Ordinalzahlen

Erste Schritte mit Superturingmaschinen
* Definition
* Beispielaufgaben
* Halteproblem
* Schwerere Aufgaben

Besondere Phänomene
* Ausbrechen aus Wiederholungen
* Lost Melody
* Abmessbare Ordinalzahlen

Effektiver Topos

"Allgemein sollten im Vortrag auch Maschinen vorgeführt werden, deren
Halteverhalten von mengentheoretischen Eigenschaften abhängt. Zum Beispiel die
Sache mit der Unabhängigkeit von BB(n) für kleine Wert von n von Adam Yedidia
und Scott Aaronson. http://www.scottaaronson.com/busybeaver.pdf"
