\documentclass[a5paper,ngerman,12pt]{scrartcl}

\usepackage[utf8]{inputenc}
\usepackage{amssymb,amsmath}

\usepackage[ngerman]{babel}
\usepackage{pdfpages}

\usepackage{graphicx}

\usepackage[protrusion=true,expansion=true]{microtype}

\usepackage{hyperref}
\usepackage[T1]{fontenc}
\usepackage{libertine}

\setlength\parskip{\medskipamount}
\setlength\parindent{0pt}

\usepackage{geometry}
\geometry{tmargin=1cm,bmargin=1cm,lmargin=1cm,rmargin=1cm}

\pagestyle{empty}

\begin{document}

\begin{center}
  \Huge\bf\sffamily
  Möbiusband-Zauberei
\end{center}

\renewcommand{\labelitemi}{$\infty$}

Für die nachfolgenden Zaubereien benötigst du jeweils ein
Möbiusband. Dazu klebst du einen Papierstreifen (diese findest du
in deinem trickreichen Mathemagiekasten) jeweils zu einem Möbiusband
zusammen, indem du je ein Ende um 180~Grad drehst und mit dem anderen Ende
verbindest. Zum Zusammenkleben kannst du normalen Leim oder Klebeband
benutzen. Und schon kann der Spaß losgehen!

\begin{itemize}
\item Das wunderliche am Möbiusband ist, dass es \emph{nur eine Seite} hat!
Lasse einen Freiwilligen mit einem Stift das Möbiusband abfahren. \emph{Ohne
abzusetzen}, wird dabei gesamte Band bemalt!

\item Wie viele Ränder besitzt das Möbiusband? Um das herauszufinden, soll ein
Frewilliger mit einem Stift die Kante des Möbiusbands abfahren.

\item Nimm ein Möbiusband und schneide es der Länge nach an der
Mittellinie entlang. Du wirst Erstaunliches feststellen! Das Band zerfällt
nicht etwa in zwei Hälften, vielmehr erhälst du ein doppelt so langes Band,
das eine ganze Umdrehung in sich hat!

\item Schneide ein Möbiusband der Länge nach scheinbar in drei
Teile. Auch hier wirst du erneut Erstaunliches feststellen. Das Band
zerfällt nun in zwei ineinander verschlungene Teile, nämlich in
ein kleines Möbiusband und ein doppelt verdrehtes Band.

\item Dieser Trick ist für Fortgeschrittene!
Du brauchst zwei \emph{entgegengesetzt verdrehte} Möbiusbänder. Klebe diese im
rechten Winkel aufeinander. Achte jedoch darauf, dass diese nicht an
der Stelle, an der die Möbiusbänder geklebt sind, aufeinander kleben. Dann
der Länge nach, wie zuvor, in der Mitte auseinanderschneiden und schön sortiert auf den
Tisch legen. Dein Publikum wird begeistert sein!
\end{itemize}

\begin{center}
  \includegraphics[scale=0.5]{herzen}
\end{center}

\newpage

\begin{center}
  \Huge\bf\sffamily
  Magie der Parität
\end{center}

Zu diesem Trick benötigst du die 36 rot-gelben Karten aus deinem
Mathemagiekasten.

Für den Mathemagietrick werden 25 Karten beliebig im Quadrat ausgelegt, wobei
die Karten auf der einen Seite rot und auf der anderen gelb sind. 
Welche Seite sichtbar sein soll, darf natürlich dein Publikum entscheiden. Nun
ergänzst du als Mathemagier -- um es schwerer zu machen -- eine sechste Reihe
rechts und unten. Danach verbindest du dir die Augen. Das Publikum darf jetzt
eine Karte umdrehen -- egal welche. Anschließend kannst durch scharfes
Hinsehen sagen, welche Karte umgedreht wurde!

Als Freund der Parität weißt du natürlich bereits wie dieser Trick
funktioniert. Die ergänzende sechste Zeile bzw. Spalte legst du entsprechend
der Parität der jeweiligen Zeile bzw. Spalte. Nachdem dein Publikum eine Karte umgedreht
hat, kannst du anhand deiner Kontrollkarten schnell sagen, welche es
war.

\end{document}
