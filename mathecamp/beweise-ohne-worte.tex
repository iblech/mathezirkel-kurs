\documentclass{../zirkelblatt}
\usepackage{floatflt}
\usepackage{geometry}
\geometry{tmargin=1.5cm,bmargin=1.5cm,lmargin=2cm,rmargin=2cm}

\let\raggedsection\centering
\pagestyle{empty}
\begin{document}

\section*{Satz des Pythagoras}
\begin{floatingfigure}[r]{4cm}
  \vspace{-1cm}
  \includegraphics[scale=0.5]{pythagoras-1}
\end{floatingfigure}
Der \emph{Satz des Pythagoras} besagt: Errichtet man auf den drei Kanten eines
rechtwinkligen Dreiecks jeweils ein Quadrat, so sind die beiden kleineren
Quadrate zusammengenommen genauso groß wie das größte Quadrat (siehe Skizze
rechts). Als Formel:
\[ a \cdot a + b \cdot b = c \cdot c. \]
\emph{Wieso ist das so?} Das sollen die beiden anderen Skizzen beantworten.
Kannst du diesen Beweis erklären?
\begin{center}
\includegraphics{pythagoras-2}
\end{center}

\vfill

\section*{Summe der natürlichen Zahlen I}
Was ist $1 + 2 + 3 + 4$? Was ist $1 + 2 + 3 + 4 + 5 + 6 + 7 + 8$? Das zu
berechnen, wird immer mühsamer. Zum Glück gibt es eine einfache Formel, die das
Ergebnis sofort liefert:
\begin{align*}
  1 + 2 + 3 + 4 + 5 + 6 + 7 + 8 &= 8 \cdot 9 : 2 \\
  1 + 2 + 3 + 4 + \cdots + 97 + 98 + 99 + 100 &= 100 \cdot 101 : 2
\end{align*}
Die Formel funktioniert auch mit jeder anderen Obergrenze als $100$. \emph{Wieso
stimmt die Formel?} Das soll die Skizze beantworten. Bei ihr ist die
Obergrenze~$6$. Kannst du den Beweis erklären?
\begin{center}
\includegraphics[scale=0.3]{kleiner-gauss-1}
\end{center}


\section*{Summe der natürlichen Zahlen II}
Was ist $1 + 2 + 3 + 4$? Was ist $1 + 2 + 3 + 4 + 5 + 6 + 7 + 8$? Das zu
berechnen, wird immer mühsamer. Zum Glück gibt es eine einfache Formel, die das
Ergebnis sofort liefert:
\begin{align*}
  1 + 2 + 3 + 4 + 5 + 6 + 7 + 8 &= 8 \cdot 9 : 2 \\
  1 + 2 + 3 + 4 + \cdots + 97 + 98 + 99 + 100 &= 100 \cdot 101 : 2
\end{align*}
Die Formel funktioniert auch mit jeder anderen Obergrenze als $100$. \emph{Wieso
stimmt die Formel?} Das soll die Skizze beantworten. Bei ihr ist die
Obergrenze~$100$. Kannst du den Beweis erklären?
\begin{center}
\includegraphics[scale=0.7]{kleiner-gauss-2}
\end{center}


\vfill
\section*{Summe der ungeraden Zahlen}
Was ist $1 + 3 + 5 + 7 + 9$? Was ist $1 + 3 + 5 + 7 + 9 + 11 + 13 + 15$? Das zu
berechnen, wird immer mühsamer. Zum Glück gibt es eine einfache Formel, die das
Ergebnis sofort liefert:
\begin{align*}
  1 + 3 + 5 + 7 + 9 + 11 \phantom{{} + 13 + 15} &= 6 \cdot 6 = 36 \\
  1 + 3 + 5 + 7 + 9 + 11 + 13 \phantom{{} + 15} &= 7 \cdot 7 = 49 \\
  1 + 3 + 5 + 7 + 9 + 11 + 13 + 15 &= 8 \cdot 8 = 64
\end{align*}
\emph{Wieso stimmt die Formel?} Das soll die Skizze beantworten. Kannst du
diesen Beweis erklären?
\begin{center}
\includegraphics[scale=0.4]{summe-ungerade-zahlen}
\end{center}


\vfill
\section*{Summe der Fibonacci-Zahlen}


\vfill
\newpage
\section*{Ein Kästchen verschwindet}
\begin{floatingfigure}[r]{3.5cm}
  \vspace{-0.3cm}
  \scalebox{0.4}{\input{flaecheninhalt-dreieck.pspdftex}}
\end{floatingfigure}
Der Flächeninhalt eines Rechtecks mit den Seitenlängen~$a$ und~$b$ ist~$a \cdot
b$. Der Flächeninhalt eines rechtwinkligen Dreiecks mit Grundseite~$a$ und
Höhe~$h$ ist~$a \cdot h : 2$ (siehe Skizze rechts). Diese beiden Formeln helfen
dir vielleicht für die Aufgabe (oder auch nicht, es gibt mehrere Lösungswege).

In den unteren beiden Skizzen ist irgendwo der Wurm drin. Denn die beiden
Figuren scheinen \emph{denselben Flächeninhalt} zu haben, doch die linke
scheint offensichtlich aus einem Kästchen mehr als die rechte zu bestehen!
Kannst du erklären, was schief läuft?

\begin{center}
  \includegraphics[scale=0.8]{ein-kaestchen-verschwindet-1}
  \hfill
  \includegraphics[scale=0.8]{ein-kaestchen-verschwindet-2}
\end{center}

\end{document}

http://upload.wikimedia.org/wikipedia/commons/d/d2/Pythagorean.svg
http://jwilson.coe.uga.edu/EMT668/emt668.student.folders/HeadAngela/essay1/image1.gif
Proofs without Words, Seite 69
http://proofsfromthebook.com/wp-content/uploads/2013/01/sum-of-first-n-positive-integers.png
http://i.stack.imgur.com/bLWK1.png
