\documentclass[12pt,compress,ngerman,utf8,t]{beamer}
\usepackage[ngerman]{babel}
\usepackage{calc}
\usepackage{ragged2e,wasysym,multicol,mathtools,tikz,txfonts,ifthen}
\usepackage[all]{xy}
\usetikzlibrary{calc,shapes.callouts,shapes.arrows}
\usepackage[protrusion=true,expansion=true]{microtype}
\hypersetup{colorlinks=true}

\graphicspath{{images/}}

\title[P vs. NP]{$\varheartsuit$ P vs. NP $\varheartsuit$}
\author[Ingo Blechschmidt]{\scriptsize\textcolor{white}{
\vspace*{-1em} \\
\textbf{36th Chaos Communication Congress} \\
\emph{Questions are very much welcome! Please interrupt me mid-sentence.} \\
\medskip
Ingo Blechschmidt}}

%\usetheme{Warsaw}
\useinnertheme[shadow=false]{rounded}
\useoutertheme{split}
\usecolortheme{orchid}
\usecolortheme{whale}
\setbeamerfont{block title}{size={}}

\useinnertheme{rectangles}

\usecolortheme{seahorse}
\definecolor{mypurple}{RGB}{150,0,255}
\setbeamercolor{structure}{fg=mypurple}
\definecolor{myred}{RGB}{150,0,0}
\definecolor{darkred}{RGB}{240,0,0}
\setbeamercolor*{title}{bg=myred,fg=white}
\setbeamercolor*{titlelike}{bg=myred,fg=white}

\usefonttheme{serif}
\usepackage[T1]{fontenc}
\usepackage{libertine}

\renewcommand{\_}{\mathpunct{.}\,}
\renewcommand{\P}{\text{P}}
\newcommand{\NP}{\text{NP}}
\newcommand{\NPC}{\text{NP{-}C}}
\newcommand{\PSPACE}{\text{PSPACE}}
\newcommand{\PSPACEC}{\text{PSPACE{-}C}}
\newcommand{\EXP}{\text{EXP}}
\newcommand{\BB}{\mathbb{B}}
\newcommand{\M}{\mathcal{M}}
\newcommand{\R}{\mathrm{R}}
\newcommand{\NN}{\mathbb{N}}
\newcommand{\RR}{\mathbb{R}}
\newcommand{\Eff}{\mathrm{Eff}}
\newcommand{\TM}{\mathrm{TM}}
\newcommand{\STM}{\mathrm{STM}}
\newcommand{\RW}{\mathrm{RW}}
\newcommand{\lambdaC}{\lambda\mathrm{C}}
\newcommand{\PA}{\mathrm{PA}}
\newcommand{\goedel}[1]{\ulcorner #1 \urcorner}
\newcommand{\Prov}{\mathrm{Prov}}
\newcommand{\True}{\mathrm{True}}
\newcommand{\Con}{\mathrm{Con}}
\newcommand{\proves}{\vdash}
\newcommand{\defeq}{\vcentcolon=}

\newcommand{\pointthis}[4]{%
  \tikz[remember picture,baseline]{\node[anchor=base,inner sep=0,outer sep=0]%
    (#3) {#3};\node[overlay,rectangle callout,%
      callout relative pointer={(#1)},fill=blue!20] at
        ($(#3.north)+(#2)$) {#4};}}

\newcommand{\kasten}[1]{%
  \setlength{\fboxrule}{2pt}%
  \setlength{\fboxsep}{8pt}%
  {\usebeamercolor[fg]{item}\fbox{\usebeamercolor[fg]{normal text}\parbox{0.2cm}{#1}}}}%

\newcommand{\slogan}[1]{%
  \begin{center}%
    \setlength{\fboxrule}{2pt}%
    \setlength{\fboxsep}{8pt}%
    {\usebeamercolor[fg]{item}\fbox{\usebeamercolor[fg]{normal text}\parbox{0.88\textwidth}{#1}}}%
  \end{center}%
}

\newcommand{\code}[1]{%
  \begin{center}%
    \setlength{\fboxrule}{1pt}%
    \setlength{\fboxsep}{8pt}%
    {\fbox{\parbox{0.81\textwidth}{#1}}}%
  \end{center}%
}

\setbeamertemplate{navigation symbols}{}

\setbeamertemplate{title page}[default][colsep=-1bp,rounded=false,shadow=false]
\setbeamertemplate{frametitle}[default][colsep=-2bp,rounded=false,shadow=false,center]

\newcommand{\hil}[1]{{\usebeamercolor[fg]{item}{\textbf{#1}}}}
\newcommand{\bad}[1]{\textbf{\textcolor{darkred}{#1}}}
\setbeamertemplate{headline}{}
\setbeamertemplate{frametitle}{%
  \vskip0.6em%
  \leavevmode%
  \begin{beamercolorbox}[dp=1ex,center]{}%
      \usebeamercolor[fg]{item}{\textbf{\textsf{\Large \insertframetitle}}}
  \end{beamercolorbox}%
}

\setbeamertemplate{footline}{%
  \leavevmode%
  \hfill%
  \begin{beamercolorbox}[ht=2.25ex,dp=1ex,right]{}%
    \usebeamerfont{date in head/foot}
    \insertframenumber\,/\,\inserttotalframenumber\hspace*{1ex}
  \end{beamercolorbox}%
  \vskip0pt%
}

\newcommand{\backupstart}{
  \newcounter{framenumberpreappendix}
  \setcounter{framenumberpreappendix}{\value{framenumber}}
}
\newcommand{\backupend}{
  \addtocounter{framenumberpreappendix}{-\value{framenumber}}
  \addtocounter{framenumber}{\value{framenumberpreappendix}}
}

\newcommand{\portrait}[4]{\begin{column}{#3\textwidth}\centering\includegraphics[height=#4\textheight]{#1}\\{\scriptsize #2\par}\end{column}}

\setbeameroption{show notes}

\input{images/primes}

\begin{document}

\addtocounter{framenumber}{-1}

{\usebackgroundtemplate{\begin{minipage}{\paperwidth}\vspace*{0.3cm}\centering\scriptsize\sieve{25}{2}\\\vspace*{3.95cm}\includegraphics[width=\paperwidth]{sun3}\end{minipage}}
\begin{frame}[c]
  \centering

  \bigskip
  \bigskip
  \bigskip
  \bigskip
  \bigskip
  \bigskip
  \bigskip
  \bigskip
  \Large

  \hil{$\varheartsuit$ P vs. NP $\varheartsuit$}
  \medskip

  \normalsize
  the biggest open question in computer science
  \bigskip
  
  \footnotesize
  \textit{-- an invitation --}
  \bigskip
  \bigskip
  \medskip

  \textcolor{black}{
    \textbf{37th Chaos Communication Congress} \\
    \emph{Questions are very much welcome! Please interrupt me mid-sentence.}
  }
  \bigskip
  \bigskip

  \tiny
  \textcolor{white}{
    Ingo Blechschmidt
  }

  \par
\end{frame}}


% \begin{document}

\begin{frame}{The landscape of complexity classes}
  \justifying
  \textbf{Def.} An algorithm~$A$
  \hil{runs in polynomial time} if and only if there is some polynomial~$p$
  such that, for every input~$I$
  \[ \text{number of steps for computing~$A(I)$} \quad\leq\quad p(|I|), \] where~$|I|$ is
  the length of an encoding of~$I$ in bits.
  \medskip
  \pause

  \begin{columns}
    \begin{column}{0.42\textwidth}
      \justifying
      \textbf{Def.} A problem is \hil{in P} if and only if there is
      is a decision algorithm which \hil{runs in polynomial time}.
      \medskip

      \textbf{Ex.} Primality testing, node reachability, \ldots
    \end{column}

    \pause
    \begin{column}{0.57\textwidth}
      \justifying
      \textbf{Def.} A problem is \hil{in NP} if and only if
      there is an algorithm which verifies \hil{wannabe certificates for a positive
      answer} in polynomial time.
      \medskip

      \textbf{Ex.} 3SAT, Sudoku, TSP, graph coloring, proof search, \ldots
    \end{column}
  \end{columns}
  \medskip

  \pause
  \textbf{Prop.} Every~$\P$-problem is also in~$\NP$: $\P \subseteq \NP$.
\end{frame}

\begin{frame}{Further complexity classes}
  \vspace*{-1.6em}\justifying
  \[ \P \ \subseteq
  \begin{array}[t]{@{}c@{}}\NP\\\rotatebox{90}{$\subseteq$}\\\NPC\end{array} \subseteq
  \begin{array}[t]{@{}c@{}}\PSPACE\\\rotatebox{90}{$\subseteq$}\\\PSPACEC\end{array} \subseteq\
  \EXP \]

  A problem~$T$ is in \ldots
  \begin{itemize}\justifying
    \item \hil{$\P$} iff there is a \hil{polynomial-time} decision algorithm.
    \item \hil{$\NP$} iff there is a polynomial-time algorithm which verifies wannabe
    certificates for a positive answer.
    \pause
    \item \hil{$\NPC$} iff it is in~$\NP$ and if
    every~$\NP$-problem is \hil{reducible} to~$T$ in polynomial time.
    \pause
    \item \mbox{\hil{$\PSPACE$} iff there is a \hil{polynomial-space} decision
    algorithm.}
    \pause
    \item \hil{$\PSPACEC$} iff it is in~$\PSPACE$ and if
    every~$\PSPACE$-problem is reducible to~$T$ in polynomial time.
    \pause
    \item \hil{$\EXP$} iff there is an \hil{exponential-time} decision algorithm.
  \end{itemize}
  \pause

  $\P \neq \EXP$, hence $\P \neq \NP$ or $\NP \neq \PSPACE$ or
  $\PSPACE \neq \EXP$.
\end{frame}

\begin{frame}{The relativization barrier}
  \justifying
  Let~$B$ be a problem. A \hil{$\boldsymbol{B}$-algorithm} is an algorithm which has access
  to an \hil{oracle for~$\boldsymbol{B}$}.
  \bigskip
  \pause

  \begin{columns}
    \begin{column}{0.36\textwidth}\justifying
      \textbf{Def.} A problem is in~$\P^B$ iff there is a polynomial-time decision
      $B$-algorithm.
    \end{column}
    \begin{column}{0.6\textwidth}\justifying
      \textbf{Def.} A problem is in~$\NP^B$ iff there is a polynomial-time
      $B$-algorithm which verifies wannabe certificates for a positive answer.
    \end{column}
  \end{columns}
  \bigskip
  \pause

  \textbf{Prop.} $\P^B \subseteq \NP^B \subseteq \PSPACE^B$.
  \pause

  \textbf{Prop.} If~$B$ is in~$\NPC$, then~$\NP \subseteq \P^B$.
  \bigskip
  \pause

  \textbf{Thm.} % (relativization barrier) \\
  For some~$B$, $\P^B = \NP^B$;
  and for some~$B$, $\P^B \neq \NP^B$.
  \pause

  \textbf{Proof, first part.} Pick for~$B$ some problem in~$\PSPACEC$. Then
  $\PSPACE \subseteq P^B \subseteq \NP^B \subseteq \PSPACE^B \subseteq
  \PSPACE$.
  % Für B in NP-C klappt es fast auch: Denn NP ⊆ P^B ⊆ NP^B. Aber NP^B ⊆ NP
  % gilt nicht!
  \pause

  \textbf{Proof, second part.} Pick for~$B$ a zero/one \hil{random oracle}. Then the problem
  ``do~$n$ consecutive ones occur in the first~$2^n$ drawings of~$B$?'' is
  in~$\NP^B$ but not in~$\P^B$.
\end{frame}

\end{document}


Ideen fürs nächste Mal:

* Mit konkreten Beispielproblemen starten
* Am Ende mehr Zeit für die Relativisierungsbarriere haben
* Folie zu wie "P = NP" aussehen könnte
* Grafik aller Komplexitätsklassen einbauen:
  https://mathoverflow.net/questions/159468/how-many-complexity-classes-do-you-know/159469#159469

* Subset Sum Problem
* Algorithmus, der in Polynomialzeit läuft, sofern P = NP gilt:
  https://en.wikipedia.org/wiki/P_versus_NP_problem#Polynomial-time_algorithms
  (Levin's universal algorithm)


Unabhängig von ZFC?

1. Frage ist äquivalent zur Existenz eines P-Entscheiders für 3SAT.

   Logische Form:
     ∃Algorithmus. ∃Polynom. ∀3SAT-Instanz.
       (Terminierung innerhalb Zeitvorgabe und Ausgabe korrekt).

   Also Σ₂-Aussage.

2. Siehe Seite 28 von Scott.

3. Jedenfalls ist P = NP also eine arithmetische Frage und damit absolut
   zwischen V und L, d.h. AC und GCH wären mit der L-Übersetzung eliminierbar.
   LEM aber nicht unbedingt.

4. LEM wäre aber aus Korrektheitsbeweisen eines gegebenen 3SAT-Entscheiders
   (mit ebenfalls gegebener polynomieller Laufzeitschranke) eliminierbar,
   da dann nur noch eine Π₁-Aussage übrig bleibt.

5. P ≠ NP, stark negiert als ∀Algorithmen. ∀Polynome. ∃3SAT-Instanz.
   (Terminierung nicht innerhalb Zeitvorgabe oder Ausgabe inkorrekt),
   ist eine arithmetische Π₂-Aussage, aus ZFC-Beweisen ließen sich also
   IZF-Beweise machen.


EXP und nicht in P: Ist es NICHT der Fall, dass eine gegebene Turingmaschine,
gegeben durch ein binäres Wort w, auf der Eingabe von w in höchstens 2^|w|
Schritten mit der Ausgabe "ja" hält?

Lässt sich in Exponentialzeit durch Simulation entscheiden, braucht etwa 2^|w|
Schritte.

Und geht nicht in Polynomialzeit: Angenommen doch, d.h. angenommen es gäbe
einen Entscheider E mit polynomieller Laufzeit p für sie. Dann finden wir eine
Kodierung w von E mit 2^|w| ≥ p(|w|). Das ist möglich, wenn wir unser
Kodierungssystem so arrangieren, dass Slack möglich ist.

Nach Voraussetzung terminiert die durch w gegebene Turingmaschine bei Eingabe
von w in höchstens p(|w|) Schritten. Wenn die Ausgabe "ja" lautet, dann
bräuchte diese Maschine aber mehr als 2^|w| Schritte, Widerspruch. Wenn die
Ausgabe "nein" lautet, dann würde sie doch "ja" ausgeben, Widerspruch.
https://www.informatik.uni-bremen.de/tdki/lehre/ss09/kt/Kapitel6.pdf


Ein Entscheidungsproblem A ist in NP genau dann, wenn eine Maschine M
existiert, die auf jeder Eingabe in Polynomialzeit hält, sodass A eine positive
Lösung genau dann hat, wenn es eine Eingabe für M gibt, mit der M mit "ja"
terminiert.

Denn: Wenn ein nichtdeterministischer Entscheider E für A mit
Polynomiallaufzeitschranke p existiert, dann nimm für M eine Maschine, die die
Eingabe als Ablaufpfad für E interpretiert und gemäß diesem E simuliert und
genau dann "ja" ausgibt, wenn diese Simulation nach der durch p vorhergesagten
Anzahl Schritte mit "ja" endet.

Umgekehrt: Ist so eine Maschine M gegeben, dann simuliere sie und entscheide
nichtdeterministisch, immer wenn sie versucht, ein Bit vom Band zu lesen. Das
wird sie höchstens polynomiell oft machen.
