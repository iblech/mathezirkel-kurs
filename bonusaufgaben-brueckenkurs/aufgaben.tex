\documentclass{../zirkelblatt1415}

\let\raggedsection\centering

\definecolor{shadecolor}{rgb}{1,1,1}

\begin{document}

\pagestyle{empty}
\enlargethispage{4.5cm}
\vspace*{-2.5cm}

\section*{Gelangweilt?}
\subsection*{Harte Bonusaufgaben zum Knobeln.}

\begin{aufgabe}{Der binomische Lehrsatz}
Zeige mit Induktion über~$n$, dass für alle Zahlen~$a$ und~$b$ gilt:
\[ (a + b)^n =
  \binom{n}{0} a^n +
  \binom{n}{1} a^{n-1} b +
  \binom{n}{2} a^{n-2} b^2 + \cdots +
  \binom{n}{n-1} a b^{n-1} +
  \binom{n}{n} b^n. \]
\emph{Tipp:} $\binom{n+1}{k} = \binom{n}{k-1} + \binom{n}{k}$.
\end{aufgabe}

\begin{aufgabe}{Kettenbruchentwicklung des goldenen Schnitts}
\label{aufg:phi}
Der \emph{goldene Schnitt} ist die Zahl~$\Phi = \frac{1 + \sqrt{5}}{2}$. Als
Teilungsverhältnis kommt~$\Phi$ an erstaunlich vielen Orten in der Natur und in
der Kunst vor. Zeige:
\[ \Phi = 1 + \cfrac{1}{1 + \cfrac{1}{1 + \cfrac{1}{1 + \ddots}}}. \]
\emph{Hinweis:} Du darfst voraussetzen, dass der Ausdruck auf der rechten Seite
überhaupt Sinn ergibt, d.\,h. konvergiert. Wie man das beweist, lernt man in
Analysis~I.
\end{aufgabe}

\begin{aufgabe}{Kettenbruchentwicklung von~$\sqrt{2}$}
Zeige:
\[ \sqrt{2} = 1 + \cfrac{1}{2 + \cfrac{1}{2 + \cfrac{1}{2 + \ddots}}}. \]
\emph{Hinweis:} Du darfst wieder die Konvergenz des
unendlichen Kettenbruchs voraussetzen.
\end{aufgabe}

\begin{aufgabe}{Die 10-adischen Zahlen}
{\scriptsize
Bei den gewöhnlichen reellen Zahlen kommen in ihrer Dezimalschreibweise vor dem
Komma nur endlich viele Ziffern, hinter dem Komma aber gelegentlich unendlich
viele Ziffern vor. Bei den~$10$-adischen Zahlen ist es genau umgekehrt: Vor dem
Komma dürfen unendlich viele Ziffern stehen, hinter dem Komma dagegen nur
endlich viele. Die Rechenverfahren zur Addition, Subtraktion und
Multiplikation, wie man sie aus der Schule kennt, funktionieren weitestgehend
unverändert. Die Division wird etwas komplizierter.\par}
\begin{enumerate}
\item Vollziehe folgende Rechnung nach:
$\ldots 13562 + \ldots 90081 = \ldots 03643$.
\item Was ist $\ldots 99999 + 1$? Dabei ist~$1 = \ldots 00001$.
\item Was ist~$-123$?
\item Finde eine~$10$-adische Zahl~$x$ -- weder Null noch Eins -- mit der
besonderen Eigenschaft~$x^2 = x$.
\end{enumerate}
\scriptsize
\emph{Bemerkung:} Die Gleichung in Teilaufgabe~c) kann man zu~$x \cdot (x-1) =
0$ umstellen. In den~$10$-adischen Zahlen kann also ein Produkt Null sein, ohne
dass einer der Faktoren Null ist. Wegen dieser schlechten Eigenschaft werden
die~$10$-adischen Zahlen kaum verwendet. \emph{Allerdings:} Verwendet man als
Basis nicht~$10$, sondern eine Primzahl, so gibt es dieses Problem nicht.
Die~$2$-adischen Zahlen werden gelegentlich in der Informatik und
die~$p$-adischen Zahlen, wobei~$p$ irgendeine Primzahl ist, überall in der
Zahlentheorie verwendet. Dort gibt es beispielsweise folgendes mächtiges
Prinzip: Eine Gleichung einer bestimmten Art hat genau dann eine Lösung
in~$\mathbb{Z}$, wenn sie eine Lösung in~$\mathbb{R}$ und in allen~$p$-adischen
Zahlen hat.\par
\end{aufgabe}

\end{document}
